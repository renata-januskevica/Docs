

\chapter{1. pielikums. Atmiņas noplūde}

    
   
\begin{lstlisting}[language=C++]
#include <stdio.h>
#include <stdlib.h>
#include <unistd.h>
#include <memory.h>

int main(int argc, char** argv)
{
	int num_allocations = atoi(argv[1]);
	int alloc_size      = atoi(argv[2]);
	printf ("Going to allocate %i chunks, %i KiB each, %i KiB in total\n",
	         num_allocations, alloc_size, num_allocations*alloc_size);

	char** arr = new char* [num_allocations];
	for (int i=0; i<num_allocations; i++)
		arr[i] = new char[alloc_size * 1024];
	printf ("Allocation done\n");

	for (int i=0; i<num_allocations; i++)
		memset (arr[i], 0, alloc_size*1024);
	printf ("Filled with data\n");

	for (int i=0; i<num_allocations; i = i+2)
		arr[i] = NULL;
        
	printf ("%i KiB are lost\n", 
			num_allocations*alloc_size);
	abort();
	delete[] arr;
} 
\end{lstlisting}

\newpage
\chapter{2. pielikums. Maksimālās atmiņas izmantošanas problēma}


\begin{lstlisting}[language=C++]
#include <stdio.h>
#include <stdlib.h>
#include <unistd.h>
#include <memory.h>

int main(int argc, char** argv)
{
	int num_allocations = atoi(argv[1]);
	int alloc_size      = atoi(argv[2]);
	printf ("Going to allocate %i chunks, %i KiB each, %i KiB in total\n",
	         num_allocations, alloc_size, num_allocations*alloc_size);

	char** arr = new char* [num_allocations];
	for (int i=0; i<num_allocations; i++)
		arr[i] = new char[alloc_size * 1024];
	printf ("Allocation done\n");

	for (int i=0; i<num_allocations; i++)
		memset (arr[i], 0, alloc_size*1024);
	printf ("Filled with data\n");

	for (int i=0; i<num_allocations-1; i++)
		delete[] arr[i];
	printf ("Released all but the last chunk is not released\n");

	abort();
} 
\end{lstlisting}

\newpage
\chapter{3. pielikums. Fragmentēšana}

\begin{lstlisting}[language=C++]
#include <stdio.h>
#include <stdlib.h>
#include <unistd.h>
#include <memory.h>

int main(int argc, char** argv)
{
	int num_allocations = atoi(argv[1]);
	int alloc_size      = atoi(argv[2]);
	printf ("Going to allocate %i chunks, %i KiB each, %i KiB in total\n",
	         num_allocations, alloc_size, num_allocations*alloc_size);

	char** arr = new char* [num_allocations];
	for (int i=0; i<num_allocations; i++) {
		if (i%2 == 0) {
			arr[i] = new char[alloc_size];
			memset (arr[i], 0, alloc_size);
		} else {
			arr[i] = new char[alloc_size + alloc_size];
			memset (arr[i], 0, alloc_size + alloc_size);
		}
	}
	printf ("Allocation done\n");

	for (int i=0; i<num_allocations; i = i+2)
		delete[] arr[i];
	printf ("Released all\n");

	abort();
} 
\end{lstlisting}

