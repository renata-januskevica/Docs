

\chapter[1. pielikums. Atmiņas noplūde] {1. pielikums \\  Atmiņas noplūde}

    
   
\begin{lstlisting}[language=C++]
#include <stdio.h>
#include <stdlib.h>
#include <unistd.h>
#include <memory.h>

int main(int argc, char** argv)
{
	int num_allocations = atoi(argv[1]);
	int alloc_size      = atoi(argv[2]);
	printf ("Tiks iedalīti %i gabali, %i KiB katrs, %i KiB kopumā.\n",
	         num_allocations, alloc_size, num_allocations*alloc_size);

	char** arr = new char* [num_allocations];
	for (int i=0; i<num_allocations; i++)
		arr[i] = new char[alloc_size*1024];
	printf ("Iedalīšana notika.\n");

	for (int i=0; i<num_allocations; i++)
		arr[i] = NULL;

	printf ("%i KiB pazaudēti.\n", 
	         num_allocations*alloc_size);

	abort();
	delete[] arr;
} 
\end{lstlisting}

\newpage
\chapter[2. pielikums. Maksimālās atmiņas izmantošanas problēma] {2. pielikums \\  Maksimālās atmiņas izmantošanas problēma}



\begin{lstlisting}[language=C++]
#include <stdio.h>
#include <stdlib.h>
#include <unistd.h>
#include <memory.h>

int main(int argc, char** argv)
{
	int num_allocations = atoi(argv[1]);
	int alloc_size      = atoi(argv[2]);
	printf ("Tiks iedalīti %i gabali, %i KiB katrs, %i KiB kopumā.\n",
	         num_allocations, alloc_size, num_allocations*alloc_size);

	char** arr = new char* [num_allocations];
	for (int i=0; i<num_allocations; i++)
		arr[i] = new char[alloc_size*1024];
	printf ("Iedalīšana notika.\n");

	for (int i=0; i<num_allocations; i++)
		memset (arr[i], 7, alloc_size*1024);
	printf ("Atmiņa ir aizpildīta.\n");

	for (int i=0; i<num_allocations-1; i++)
		delete[] arr[i];
	printf ("Atmiņa ir atbrīvota izņemot pēdējo gabalu.\n");

	abort();
	delete[] arr;
} 
\end{lstlisting}

\newpage
\chapter[3. pielikums. Fragmentēšana] {3. pielikums \\  Fragmentēšana}


\begin{lstlisting}[language=C++]
#include <stdio.h>
#include <stdlib.h>
#include <unistd.h>
#include <memory.h>

int main(int argc, char** argv)
{
	int num_allocations = atoi(argv[1]);
	int alloc_size      = atoi(argv[2]);
	printf ("Tiks iedalīti %i gabali, %i baitu katrs, %i baitu kopumā.\n",
	         num_allocations, alloc_size, num_allocations*alloc_size);

	char** arr = new char* [num_allocations];
	for (int i=0; i<num_allocations; i++) {
		if (i%2 == 0) {
			arr[i] = new char[alloc_size];
			memset (arr[i], 0, alloc_size);
		} else {
			arr[i] = new char[alloc_size+16];
			memset (arr[i], 0, alloc_size+16);
		}
	}
	printf ("Iedalīšana notika.\n");
	printf ("Atmiņa ir aizpildīta.\n");

	for (int i=0; i<num_allocations; i = i+2)
		delete[] arr[i];
	printf ("Atmiņa uz kuru norāda katra otra norāde ir atbrīvota.\n");

	abort();
	delete[] arr;
} 
\end{lstlisting}

\newpage
\chapter[4. pielikums. Gdb skripts atmiņas noplūdes atkļūdošanai] {4. pielikums \\  Gdb skripts atmiņas noplūdes atkļūdošana}
\lstinputlisting [language=Python]{leak.gdb}

\newpage
\chapter[5. pielikums. Gdb skripts pārējo divu problēmu atkļūdošanai] {5. pielikums \\ Gdb skripts fragmentēšanas un maksimālās atmiņas izmantošanas problēmas atkļūdošanai}
\lstinputlisting [language=Python]{fragmentation.gdb}
