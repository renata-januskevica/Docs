Darbam ir šādi galvenie rezultāti:
\begin{itemize}
    \item Pētījuma rezultātā tika izstrādāta kaudzes atkļūdošanas metode, kura balstīta uz atmiņas izmetes analīzi.
    Metode tika nodemonstrēta darbībā, izmantojot trīs analizatoru piemērus.
    Pēc katras analizatoru sniegtās izdrukas bija iespējams identificēt vienu no trim pētāmām kaudzes problēmām.
    \item Lai atrastu problēmām raksturīgas pazīmes atmiņas izmetē, katrai pētāmai problēmai tika uzģenerēta atmiņas izmete.
    Katrs atmiņas izmetes saturs tika izpētīts sīkāk ar gdb atkļūdotāja palīdzību, rezultātā tika identificētas pazīmes izvēlētajām problēmām.
    \item Tika izstrādāti un aprakstīti trīs analizatoru piemēri: atmiņas noplūdes analizators, maksimālās atmiņas izmantošanas problēmas analizators, fragmentēšanas problēmas analizators.
    Analizatori tika izstrādāti divos gdb skriptos un veic savu galveno uzdevumu: sniedz izdruku par identificētām problēmu pazīmēm. 
    \item Darbā katrai pētāmai kaudzes problēmai aplūkoti problēmu veidi, cēloni un izpausme strādājošā sistēmā.
\end{itemize}
%Atmiņas izmete var tikt pielietota kaudzes atkļūdošanai. %Tā kā eksistē vēl divas zināmas kaudzes problēmas, tad

Balstoties uz galvenajiem darba rezultātiem var secināt, ka pieeja strādā un var tikt izmantota plašāk, piemēram, citu problēmu atkļūdošanai.
Viecot pētījumu autore ieguva zināšanas par kaudzes problēmām, to pazīmēm un iedalītāja realizāciju.
Zināšanas turpmāk tiks pielietotas ikdienas darbā un atvieglos programmas atkļūdošanas procedūru problēmām, kas saistītas ar kaudzi un dinamisko atmiņas iedalīšanu. 

Turpmāk iespējams turpināt darbu dotajā virzienā, jo bakalaura darbā nav izpētītas visas zināmas kaudzes problēmas un nav identificētas to pazīmes atmiņas izmetē.
Ir iespējams izveidot analizatoru, kurš sniegs kopējo statistiku pār visām zināmam kaudzes problēmām un strādās neatkarīgi no atkļūdotāja. 
Tā kā kaudzes problēmām var nebūt tiešo pazīmju, tad šīm analizatoram būs savs pielietojums, jo šobrīd tāds analizatora risinājums neeksistē.
