Bezvadu sensoru tīkli tiek pielietoti vairākās nozarēs. Piemēram, medicīnā, militārā, aizsardzības un kontroles nozarēs. Bezvadu sensoru tīkli var strādāt kā atsevišķs tīkls, gan arī var būt iekļauti citos tīklos. Darbā tiek piedāvāta bezvadu sensoru tīkla sistēmas arhitektūra, kas sadalīta divos līmeņos. Dota sistēma ir domāta strādāt eksistējošā TCP/IP tīkla infrastruktūrā nodrošinot bez sadursmju komunikāciju bezvadu sensoru līmenī. Piedāvāta sistēma tiek salīdzināta ar dažām līdzīgām eksistējošām sistēmām. Tiek pierādīts kā eksistējošas sistēmas neatbilst izvirzītiem ierobežojumiem. 

Pirmā sistmas līmenī tiek izvietots klasterizēts sensoru tīkls. Šī līmeņa īpašība ir bez sadursmju komunikācija. Darbā tiek piedāvāts MAC slāņa sensoru tīkla protokols bez sadursmju vides piekļuvei. Kā arī tiek piedāvāts autonomas klasterizācijas algoritms. Pateicoties bez sadursmju īpašībai MAC protokolu izdevās uzprojektēt determinētu ar iespēju pielietot reāla laika uzdevumu risināšanai. Tika veikts MAC protokolu ar klasterizācijas algoritmiem salīdzinājums ar darbā piedāvātu pieeju. Tika secināts, ka neviens no apskatītiem risinājumiem neatbilst visām MAC protokolam izvirzītām prasībām. 

Otrais sistēmas līmenis ir domāts kā sensora tīkla izejas punkts citos tīklos. Izeju nodrošina katram sensoru klasterim pievienots vārtejas mezgls. Pateicoties transporta protokolam, vārtejas nodrošina datu nogādāšanu no sensoru tīkla, kā arī tā konfigurēšanu un uzturēšanu. Viena no vārteju īpašībām skar vairāku klasteru atbalstīšanu atbilstošu vārteju bojājuma gadījumos. Piedāvātas sistēmas izstrāde tiek veikta izstrādājot arhitektūrai atbilstošu modeli SysML valodā. Darbā tiek piedāvāta divu līmeņu sistēmas projektēšanas metodika, kas skar gan MAC protokola, gan vārtejas projektēšanu un izveidošanu. Tiek piedāvāta statistiska modeļa izmantošana divu līmeņu sistēmas aparatūras izvēlei un sistēmas veiktspējas prognozēšanai.

Darbs sastāv no ievada, 6 nodaļām, secinājumiem un 3 pielikumiem. ottab\ tabulas pamattekstā un \total{citenum} nosaukumi literatūras sarakstā.

