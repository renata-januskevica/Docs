Darbam ir šādi galvenie rezultāti:
\begin{itemize}
    \item Pētījuma rezultātā tika izstrādāta kaudzes atkļūdošanas metode, kura balstīta uz atmiņas izmetes analīzi.
    Metode tika nodemonstrēta darbībā, izmantojot trīs analizatoru piemērus.
    Pēc katras analizatoru sniegtās izdrukas bija iespējams identificēt vienu no trim pētāmām kaudzes problēmām.
    \item Lai atrastu problēmām raksturīgas pazīmes atmiņas izmetē, katrai pētāmai problēmai tika uzģenerēta atmiņas izmete.
    Katrs atmiņas izmetes saturs tika izpētīts sīkāk ar gdb atkļūdotāja palīdzību, rezultātā tika identificētas pazīmes izvēlētajām problēmām.
    \item Balstoties uz GNU C pirmkodu, literatūru ir uzzīmēta ptmalloc2 iedalītāja organizācijas reprezentācija.
\end{itemize}
%Atmiņas izmete var tikt pielietota kaudzes atkļūdošanai. %Tā kā eksistē vēl divas zināmas kaudzes problēmas, tad

Turpmāk iespējams turpināt darbu dotajā virzienā, jo bakalaura darbā nav izpētītas visas zināmas kaudzes problēmas un to pazīmes atmiņas izmetē.
Ir iespējams izveidot analizatoru, kurš sniegs kopējo statistiku pār visām zināmam kaudzes problēmām. 
Tā kā kaudzes problēmām var nebūt tiešo pazīmju, tad šīm analizatoram būtu savs pielietojums.
