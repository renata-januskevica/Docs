Šajā nodaļā tiek aplūkotas divas problēmas, kuru pētīšanai darba praktiskajā daļā tiks piedāvāta atkļūdošanas metode.

\section{Atmiņas noplūde}
Atmiņas noplūde ir viena no bieži sastopamām problēmām. 
Par atmiņas noplūdi var nosaukt procesu, kurā laikā atmiņas adreses kļūst nepieejamas, pazaudētas. To var salīdzināt ar dārglietām, kas apraktās mežā, neatzīmējot tās atrašanas vietu kartē. 
Arī atmiņā, programmētāju kļūdu dēļ, var tikt izveidots liels atmiņas apgabals bez norādēm, kurš vairāk nebūs pieejams.
Tā kā atmiņas noplūdes pazīmes bieži paliek nemanāmas \cite{surv_of_dist}, tad ir vērts rūpīgi izpētīt doto problēmu, lai zinātu problēmas cēloņus, pazīmes, pētīšanas metodikas un rīkus. 
Tāpēc šajā sadaļā tiks izklāstīta atmiņas noplūdes problēma.



%\subsection{Lietotāju atmiņas pārvaldība C un C++}



\subsection{Atmiņas noplūdes pazīmes un sekas}

Atmiņas noplūde ir problēma, kas nav vēlama sekojošos gadījumos: 
 
\begin{itemize}
\item { Programmas darbībai ir nepieciešams izdalīt daudz dinamiskās atmiņas resursu, }
\item { Servera vai citas programmas, kuras darbojās visu laiku bez apstājas, }
\item { Reālā laikā sistēmās, jo ir svarīgi iegūt rezultātu ierobežotā laikā. }
\end{itemize}

Viena no svarīgākam atmiņas noplūdes pazīmēm, ko var novērot lietotājs, ir noplūdi izraisošā procesa un pārējo procesu palēnināšana. 
Tas notiek, jo operatīvajā atmiņā pietrūkst vieta un dati šajā brīdī tiek pārnesti (swapped out) cietajā diskā.
Bet, procesa pārnešana ir laikietilpīga operācija. 
Kaut kāda brīdī, kad tiks izsmelti visi resursi, katrs malloc pieprasījums būs neveiksmīgs un sekas var būt ļoti dažādas, atkarībā no sistēmas. 
Novēršot problēmu programma kļūs pirmām kārtām uzticamāka, ātrāka un kvalitatīvāka.

\section{Datu struktūru integritātes problēma}
\section{Pētīšanas metodes}
\section{Izmantojamie rīki}
