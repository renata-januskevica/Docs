 \label{sec:problems}
 Darbā tiek aprakstītas 5 kaudzes problēmas: atmiņas noplūde, fragmentēšana, maksimālās atmiņas izmantošanas problēma, datu kaudzes bojāšana, kļūdas trešās puses bibliotēkās.
3 no 5 problēmām tiek izpētītas detalizēti.
Par kaudzes problēmām tiek uzskatītas problēmas, kuras rodas lietotnē nepareizās kaudzes pārvaldības dēļ: 
\begin{itemize}
\item nekorekta kaudzes pārvaldība ar malloc() vai citām līdzīgām funkcijām;
\item nekorekta kaudzes pārvaldība, ko nodrošina iedalītājs.
\end{itemize}

\section{Atmiņas noplūde}

Atmiņas noplūde (memory leak) ir viena no bieži sastopamām problēmām C un C++ valodās \cite{GNED}.
Atmiņas noplūde notiek nepareizās lietotāja atmiņas pārvaldības dēļ, kad atmiņa, kura vairs netiks izmantota programmā, netiek atbrīvota.

Atmiņas noplūdes problēmu var sadalīt divos dažādos veidos: fiziskā un loģiskā atmiņas noplūde \cite{JMMR}.
Fiziskā atmiņas noplūde ir novērojama, kad atmiņas adreses, kuras tika iedalītas procesam,  kļūst nepieejamas, pazaudētas, tas notiek, kad procesa adrešu telpā uz iedalīto atmiņas gabalu kaudzē nenorāda neviens rādītājs.
Šīs programmas stāvoklis var būt novērojams 3 iemeslu dēļ \cite{JMMR}:
\begin{itemize}
\item pēdēja norāde uz atmiņas gabalu ir pārrakstīta vai norāde bija palielināta, piemēram, lai sasniegtu datus ar nobīdi;
\item norāde atrodas ārpus darbības lauka (out of scope);
\item atmiņas bloks, kurš glabāja norādi, bija atbrīvots.
\end{itemize}

Loģiskā atmiņas noplūde ir novērojama, kad iekšējā buferī, rindā vai citā datu struktūrā ir uzglabātas norādes uz dinamiski iedalītu atmiņu, bet norāžu skaits pieaug neierobežoti.
Loģiskā atmiņas noplūdi bieži nosauc par slēpto atmiņas noplūdi (hidden memory leak) \cite{RRUU}, jo atmiņa ir joprojām sasniedzama no programmas, bet nekad netiek atbrīvota.

Abos gadījumos sekas ir vienādas.
Sākumā tiks novērota pakāpeniskā procesa palēnināšana, jo daļa no informācijas tiks uzglabāta lapošanas failā (paging file).
Kaut kāda brīdī, kad tiks iztērēta visā dinamiskā atmiņa, katrs malloc() funkcijas izsaukums būs neveiksmīgs.
Šeit var notikt kritiskā kļūda, kuras cēlonis ir sliktā programmēšanas prakse.
Programmētāji ne vienmēr pārbauda malloc() rezultātu pirms vērsties pēc malloc() funkcijas atgrieztās norādes.  
Mēģinājums piekļūt null adresei  izraisīs Segmentation fault kļūdu.
Ja programmā bija paredzēts, ka malloc() var atgriezt null, tad process turpinas izpildi ierobežotā režīmā, jo vairs nav iespējams dinamiski iedalīt atmiņu un izpildīt daudzus uzdevumus. 
Daudzās sistēmās tas nav pieļaujams un var tikt uzstādīti dažādi ierobežojumi, kuri pēc ierobežojošās vērtības sasniegšanas (izpildes laiks, patērētās atmiņas) automātiski pārtrauks procesa darbību. 

\begin{figure}[h]
\begin{lstlisting}
#include <string>
using namespace std;

int main() {
    string *str;

    for (int i=0; i<10001; i++) {
        // 10000*14 bytes are lost
        str = new string("Hello, World!");
    }
    delete str;

    return 0;
}
\end{lstlisting}
\caption{\textbf{\fontsize{11}{12}\selectfont {Atmiņas noplūde, C++}}}
\end{figure}

Atmiņās noplūdes problēma ir uzskatāmi nodemonstrēta piemērā (sk. 3.1. attēlu).
Programma  iedala 10001 atmiņas gabalus ar new operatora palīdzību. 
Rādītājs \texttt{str} katru reizi tiek pārrakstīts un norāda uz kārtējo iedalīto atmiņas gabalu, kurā izmērs ir 14 baiti.
Tā kā atmiņas adreses kļūst pazaudētas un nav iespējas piekļūt iepriekšējiem elementiem pēc tam kad  \texttt{str} radītājs ir parakstīts, tad piemēra ir redzama fiziskā atmiņas noplūde.
Beigās tiek atbrīvots tikai viens atmiņas gabals, kurš bija iedalīts pēdējais. 
Programmas darbības laikā kļūst pazaudēti 10000 gabali, kuru kopējais izmērs ir 140000 baiti.
 Pēc programmas izpildes beigām visā procesam iedalītā atmiņa tiek atgriezta operētājsistēmai.


Sekojošos gadījumos sistēmas kļūst viegli ievainojamas, ja tajās ir kļūda, kas izraisa atmiņās noplūdi \cite{RTTV}: 
\begin{itemize}
\item { kad operētājsistēma neatbrīvo, lietotnes izpildei izmantoto atmiņu pēc tam, kad lietotne beidz savu darbību, piemēram, AmigaOS;}
\item { ja servera vai citās programmas darbojās visu laiku bez apstāšanās; }
\item { ja portatīvām ierīcēm ir ierobežots atmiņas daudzums;}
\item { ja programmas pieprasa atmiņu uzdevumiem, kuri izpildās ilgstošu laika periodu; }
\item { reālā laikā sistēmās, jo ir svarīgi iegūt rezultātu ierobežotajā laikā. }
\end{itemize}

Atmiņas noplūdes problēmu ir grūti atkļūdot,  jo nav zināmi nosacījumi, kuriem izpildoties notiek atmiņas noplūde. 
Ja ir redzamas sekas (ir atmiņas izmete un programma pabeidza savu darbību), bet nav zināms problēmas cēlonis, tad izstrādātājiem ir nepieciešams daudz resursu, lai atkārtotu un izlabotu atmiņas noplūdi. 
Eksistē vairāki rīki, kuri palīdz atkļūdot atmiņas noplūdes problēmu, tādi ka: Valgrind, Totalview, Purify. 
Taču tie ne vienmēr sniedz pietiekamu informāciju un bieži netiek izmantoti strādājošās sistēmās, jo piedāvātas atkļūdošanas tehnikas un rīki var palēnināt sistēmas darbību.
Piemēram, ieslēdzot  memcheck rīku iekš Valgrind instrumentācijas ietvara, programmas izpildes ātrums palēninās aptuveni 20-30 reizes \cite{UVD}.

Reālajās sistēmās problēma var izpausties uzreiz pēc palaišanas, bet var kļūt novērojama tikai pēc dažiem gadiem. 
Abi gadījumi ir izplatīti \cite{HTTM}.
Tā kā atmiņas noplūdes rezultātā atmiņa tiek pazaudēta, tad var periodiski novērot procesa atmiņas patēriņa pieaugumu.
Pazīme, kas varētu liecināt par atmiņas noplūdi strādājošā sistēmā ir pārmērīgs\footnote{Šajā kontekstā pārmērīgs nozīme, ka izmērs ir lielāks par to, kuru paredz programmētājs un tas rāda pamatotas šaubas, par atmiņas noplūdes problēmas esamību programmā.}
atmiņas daudzums, kas visu laiku pieaug. 
Kad process izmanto pārmērīgo atmiņu un izmantotās atmiņas daudzums nemainās, tad šī pazīme var dot tikai aptuvenu novērtējumu par dotās problēmas esamību, jo eksistē vairākas citas problēmas, piemēram, fragmentēšana, maksimālās atmiņas izmantošanas problēma vai kļūdas trešās puses bibliotēkās, kuras var palielināt izmantotās atmiņas daudzumu.


\section{Maksimālās atmiņas izmantošanas problēma}
 \label{sec:peak_mem}
 Maksimālās atmiņas izmantošanas (peak memory utilization) problēma var notikt, kad iedalītu un atbrīvotu gabalu izmēru summa kaudzē sasniedz maksimumu procesa izpildes laikā.
Ir svarīgi pievērst uzmanību gadījumiem, kad var tikt sasniegts maksimums.
Piemēram, tas var notikt, kad process tiecās pie trapa virsotnes vai maksimuma punkta.

Atmiņas daudzums, kas tiek izmantots programmas izpildes laikā var visu laiku mainīties.
Pētījumā \cite{PWMS} tiek apkopoti trīs svarīgākas atmiņas izmantošanas shēmas: traps (ramps),  maksimums (peaks),  plato (plateaus).
Citas atmiņas izmantošanas shēmas ir iespējamas, bet izpaužas ļoti reti.
Ne visām programmām ir raksturīgas visās trīs shēmas, bet vairākumam ir raksturīga viena vai divas no tām.
Šīs shēmas tika apkopotas, balstoties uz kvantitatīvo programmu novērtējumu \cite{PWMS}. 
\begin{itemize}
\item Traps. Programma uzkrāj datu struktūras monotoni. 
Tas varētu notikt, tāpēc ka uzdevuma atrisināšanai ir nepieciešams paveikt daudzas darbības un pakāpeniski uzbūvēt daudzas  datu struktūras. 
Lai atrisinātu uzdevumu, atmiņas patēriņš monotoni aug. Pēc uzdevuma atrisināšanas atmiņas patēriņš strauji samazinās;
\item Maksimums. Šo veidu var nosaukt par trapu tikai ļoti īsa laika periodā.
Daudzām programmām var būt nepieciešams izveidot lielas datu struktūras, kāda uzdevuma izpildīšanai.
Pēc šī uzdevuma pabeigšanas gandrīz visā pieprasītā atmiņa var tikt atbrīvota.
Grafiks šai shēmai izskatās kā lauztā līnija un atmiņas patēriņš var svārstīties dramatiski;
\item Plato. Novērojama, kad programmas ātri uzbūve datu struktūras un izmanto tās ilgā laika periodā, bieži izmanto līdz programmas izpildes beigām.
\end{itemize}


Problēma ir novērojama, kad liels atmiņas daudzums netiek atgriezts operētājsistēmai pēc izmantošanas, pat tad, ja gandrīz visa atmiņa tika atbrīvota ar free() vai delete palīdzību.
Rezultātā process var patērēt pārmērīgo atmiņas daudzumu, kurš nebija paredzēts projektējumā.
Šī situācija kļūst iespējama, ja notiek daudzi pieprasījumi pēc atmiņas, kas ir mazāki par 128 kilobaitiem.
Pieprasījumi pēc lielākiem atmiņas gabaliem tiks apstrādāti ar mmap() sistēmas izsaukumu un neizraisīs doto problēmu.
Pēc mmap() izsaukumiem atmiņu ir iespējams atgriezt operētājsistēmai ar munmap() palīdzību, jo atmiņa neatrodas kaudzē. 
Izmantojot brk() sistēmas izsaukumu, kamēr netiks atbrīvots atmiņas gabals, kas atrodas beigās, atmiņa netiks atgriezta operētājsistēmai.

Strādājošā sistēmā problēma ir novērojama kā pārmērīgs atmiņas patēriņš pēc trapa virsotnes vai maksimālā punkta sasniegšanas.
Turpmāk tiks apskatīts piemērs, kurš demonstrē to kā izpaužas dotā problēma.
Lai kontrolētu atmiņas patēriņu, procesa izpildes laikā, tika izmantota \texttt{ps} komanda.
Procesam patērēts atmiņas daudzums iegūts no RSS un VSZ rādītājiem.
VSZ parāda virtuālo atmiņu, RSS parāda fizisko atmiņu, kuru izmanto process.
Rādītāju mērvienība ir kilobaits.
Tika palaista programma un katrā programmas solī tika noņemti radītāji (sk. 3.1. tabulu).
Tā kā bija iedalīti 100 gabali, katrs 100 kilobaitu izmērā, tad kaudze bija paplašināta ar brk() sistēmas izsaukumu.
Kopēja pieprasīta atmiņa bija vienāda ar 10000 kilobaitiem.
Iegūtie rādītāji parāda, ka atmiņa pilnībā tika atbrīvota tikai pēc tam, kad bija atbrīvots pēdējais atmiņas gabals.
To var redzēt 4 solī, kur VSZ un RSS rādītāji paliek nemainīgi, salīdzinot ar iepriekšējo soli.
Turklāt 5 solī, pēc pēdējā gabala atbrīvošanas, var novērot to, ka atmiņās daudzums samazinās, tas ir izskaidrojams ar to, ka atmiņa tika atgriezta operētājsistēmai.


\begin{table}[H]
\caption{\textbf{\fontsize{11}{12}\selectfont {Programmas RSS un VSZ radītāji}}}
\label{table:kysymys}
\centering
	\begin{tabular}{|l|l|l|p{5cm}|}
	  \hline
	Solis & VSZ & RSS \\
    \hline
    1. Sākums & 3228 & 612 \\
	\hline
	 2. Ar new ir pieprasīti 100 gabali, katrs 100 kilobaitu izmērā  & 13360  & 1136 \\
      \hline
     3. Atmiņa ir aizpildīta ar 0 &  13360 & 10640\\
      \hline
     4. Atmiņa tiek atbrīvota izņemot pēdejo gabalu & 13360 & 10640 \\
      \hline
     5. Atmiņa tiek pilnībā atbrīvota & 3360 & 968 \\
    \hline
	\end{tabular}
\end{table}



\section{Fragmentēšana}

"Ir pierādīts, ka katram atmiņas iedalīšanas algoritmam, vienmēr ir iespējama situācija, ka kāda lietotne pieprasīs 
un atbrīvos atmiņu tāda veidā, ka tās nojauks iedalītāja stratēģiju un izraisīs lielu fragmentēšanu. 
Ir pierādīts ne tikai tas, ka nav laba iedalīšanas algoritma, bet arī tas, ka katrs iedalīšanas algoritms var būt slikts dažām lietotnēm" \cite{PWMS}.
Tātad, fragmentēšanas problēma var būt aktuālā daudzām C un C++ lietotnēm, kuras pieprasa atmiņu no kaudzes.

Fragmentēšanas problēmu var iedalīt divos dažādos veidos: iekšējā un ārējā fragmentēšana.
Iekšējā fragmentēšana notiek, kad tiek iedalīts lielāks atmiņas gabals nekā tika pieprasīts.
Izlīdzināšana ir viens no iekšējās fragmentēšanas cēloņiem.
Iekšējo fragmentēšanu ir iespējams paredzēt, jo var izskaitļot kuram skaitlim tiks noapaļots izmērs.
GNU C bibliotēkā notiek atmiņas gabalu izlidināšana 8 - ka vai 16 - reizinājumam.
Izlīdzināšana samazina atšķirīgu gabalu izmēru skaitu kaudzē.
Nodrošinot izlīdzināšanu, ir palielināta iekšējā, turklāt ir samazināta ārējā fragmentēšana \cite{RAN}.
Ārējā fragmentēšana ir nespēja iedalīt atmiņas gabalu kaudzē, kad kaudzē pietiekoši daudz brīvas atmiņas, lai apmierinātu doto pieprasījumu.
Ārējā fragmentēšana var izpausties ar laiku, kad daudzas reizes jau tika iedalīti un atbrīvoti dažāda izmēra atmiņas gabali.
Fragmentēšanas rezultātā pārmērīgi tiek izlietoti kaudzes resursi, jo kad pieprasījums pēc atmiņas nevar tikt apmierināts, tad notiek kaudzes piespiedu paplašināšana.

Fragmentēšanu mēra procentos (\%). 
Stradajošā sistēmā var būt vairāki veidi kā var mērīt atmiņas fragmentēšanu \cite{MSJ}. 
Atmiņas izmetē ir iespējams izrēķināt ārējo fragmentēšanu tikai uz procesa partraukšanas brīdī.
Tātad ir iespējams izrēkināt tikai momentāno kaudzes fragmentēšanu.
Fragmentēšana var būt izrēķināta kā attiecība starp atmiņas daudzumu kaudzē, ko aizņem iedalītājs pret atmiņas daudzumu, ko izmanto process (neietilpst atbrīvotie atmiņās gabali).

Turpmāk ir apskatīts piemērs, kurš demonstrē iekšējās fragmentēšanas cēloņi.
Šīs C valodā uzrakstītais kods izdruka atmiņas gabala izmēru, kurš īstenībā tiek iedalīts no kaudzes (sk. 3.2. attēlu).
Piemēra ir redzams, ka tiek iedalīti 4 baiti, bet programma izdruka beigu rezultātu - 16 baiti.
Dotajā piemērā iekšējā fragmentēšana ir vienāda ar 12 baitiem.

Algoritms ir sekojošs: 
\begin{enumerate}
\item ar malloc() tiek iedalīts atmiņas apgabals;
\item  tiek iegūta size elementa vērtība (objektam ar malloc\_chunk struktūru);
\item  tiek atņemtas A, M, P kontroles zīmes 111 = 7 un iegūts iedalītā atmiņas  gabala izmērs;
\item  tiek atbrīvota atmiņa.
\end{enumerate}

\begin{figure}[h]
\begin{lstlisting}
 #include <stdio.h>
 #include <malloc.h>

 int main () {
     char * ptrl;
     int chunk_size;

     ptrl = (char *)malloc(4);
 
     /* get value of chunk size (the second malloc_chunk element) */
     chunk_size = *((char *) ptrl - sizeof(size_t));
     /* the lower 3-bits are used as metadata */
     chunk_size = chunk_size - (chunk_size & 7);
 
     printf("size = %d\n", chunk_size);
     free(ptrl);
 
     return 0;
 }
\end{lstlisting}
\caption{\textbf{\fontsize{11}{12}\selectfont {Izmēra noteikšana iedalītām gabalam}}}
\end{figure}

\section{Datu kaudzes bojāšana}

Viena kļūda programmā var sabojāt kaudzes datus. 
Šo kļūdu ir grūti atrast, jo sekas ir novērojamas nevis tad, kad tiek pārrakstīti dati kaudzē, bet kad ir nākamais mēģinājums piekļūt pārrakstītiem datiem.
Kaudzē ir novērojamās vairākas bojāšanas kļūdas (heap corruption) \cite{DHC}:
\begin{itemize}
\item robežu pārpildīšana (boundary overrun). Notiek, kad programma raksta aiz malloc() funkcijas robežām;
Tāda veidā var parakstīt nākamo datu struktūru atmiņā. 
\item problēma, kad programma raksta pirms malloc() robežām;
\item piekļuve neinicializētam atmiņas gabalam. Programma mēģina lasīt datus no gabala, kurš nav inicializēts;
\item piekļuve atbrīvotām gabalam. Programma mēģina lasīt vai rakstīt atmiņas gabalā, kurš bija atbrīvots;
\item divkārša atbrīvošana. Programma atbrīvo datu struktūras, kuras jau tika atbrīvotas;
\item programma free() funkcijai padod adresi, kura nebija atgriezta ar malloc().
\end{itemize}



\section{Kļūdas trešās puses bibliotēkās}
Tā kā izstrādājama programma strādā, izmantojot trešās puses bibliotēkas, tad programmas uzticamība un kvalitāte ir atkarīga arī no iedalītāja trešās puses bibliotēkās.
Ir iespējams, ka notiek problēma programmā nekorektās iedalītāja realizācijas dēļ \cite{MUE}.
Ja iedalītais ir uzturēts, tad iespējams, ka var piereģistrēt problēmas, lai nākamajās versijās tas tiktu izlabotas.
Izmantojot individuālo iedalītāja risinājumu problēmas būs jāatkļūdo patstāvīgi.  
Daži iedalītāji var būt izstrādāti ar iepriekš zināmiem ierobežojumiem, piemēram, ja iedalītāja algoritms ir efektīvs, tad tas var izmantot pārmērīgo atmiņas daudzumu.
Vai var tikt izmantots mazs atmiņas daudzums, turklāt algoritms būs ļoti neefektīvs.
Šīs īpašības ir jāņem vērā izstrādājot lietotnes, lai nerastos problēmas, kas saistītas ar trešās puses bibliotēkām.
Ja problēma notiek iekšā trešās puses bibliotēkās, tad ir jāatrod ceļš ka nepieļaut doto kļūdu, ir jāgaida atjauninājumus, kurā kļūda tiks izlabota, vai ir jānomaina tekošais iedalītājs. 
\section{Secinājumi}
\begin{enumerate}
\item 
\end{enumerate}



