\section{Atmiņas noplūde}


Atmiņas noplūde ir lietotnēs sastopamā problēma, ko bieži mēģina atkļūdot ar dažādu rīku palīdzību \cite{PUG, VAL}. 
Atmiņas noplūde notiek, kad procesam iedalītās atmiņas adreses kļūst nepieejamas, pazaudētas.
Tas var notikt dažādu iemeslu dēļ. 


Ja atmiņa pēc izmantošanas netiek nekad atbrīvota, un katru reizi, izpildot vienu un to pašu koda gabalu, iedalīta no jauna, tad pieejams operētājsistēmai atmiņas daudzums ar laiku samazinās.
Sākumā sistēma paliek arvien lēnāka, pēc tam parasti notiek sistēmas apstāšanās.
Lielākas iespējas pieļaut atmiņas noplūdi ir C valodā. 
Tas ir atkarīgs no konkrētas realizācijas, taču C++ kodā ir iespēja izmantot konstruktorus un destruktorus, lai iedalītu un atbrīvotu atmiņu.
Šīs piegājiens ļauj izsekot atmiņas iedalīšanu un atbrīvošanu, ka arī samazina kļūdu parādīšanās varbūtību, programmām, kas izstrādātas C++ valodā.
Kaut arī eksistē valodās, kurās priekš atmiņās atbrīvošanas ir paredzēts drāzu savācējs un atbrīvošanas operācijas nav jāveic manuāli, piemēram: Java, C\#, Python un Ruby valodās. 
Taču šajās valodās arī pastāv iespēja pieļaut atmiņas noplūdi izraisošās kļūdas. Šī problēma, kaut arī ir raksturīga C un C++ valodām, ir novērojamā arī citās valodās.

Arī atmiņā, programmētāju kļūdu dēļ, var tikt izveidots liels nepieejams atmiņas apgabals bez norādēm.

\subsection{Atmiņas noplūdes sekas}

Atmiņas noplūde ir problēma, kas īpaši nav pieļaujama sekojošos gadījumos: 
 
\begin{itemize}
\item { Programmas darbībai ir nepieciešams izdalīt daudz dinamiskās atmiņas resursu, }
\item { Servera vai citas programmas, kuras darbojās visu laiku bez apstājas, }
\item { Reālā laikā sistēmās, jo ir svarīgi iegūt rezultātu ierobežotā laikā. }
\end{itemize}

Viena no svarīgākam atmiņas noplūdes pazīmēm, ko var novērot lietotājs, ir noplūdi izraisošā procesa un pārējo procesu palēnināšana. 
Tas notiek, jo uz operatīvas atmiņas pietrūkst vietas un dati šajā brīdī tiek pārnesti (swapped out) cietajā diskā.
Bet, procesa pārnešana ir laikietilpīga operācija. 
Kaut kāda brīdī tiks izsmelti visi resursi, katrs malloc pieprasījums būs neveiksmīgs un sekas var būt ļoti dažādas, tas ir atkarīgs no sistēmas. 
Novēršot problēmu programma kļūs pirmām kārtām uzticamāka, ātrāka un kvalitatīvāka.

\subsection{Atmiņas noplūdes pazīmes}



\section{Atmiņas nevienmērīga lietošana}
\subsection{Atmiņas nevienmērīga lietošanas sekas}
\subsection{Atmiņas nevienmērīga lietošanas pazīmes}

\section{Fragmentēšana}
\subsection{Fragmentēšanas sekas}
\subsection{Fragmentēšanas pazīmes}

\section{Kļūdas glibc bibliotēkā}
\subsection{Kļūdas glibc sekas}
\subsection{Kļūdas glibc pazīmes}
