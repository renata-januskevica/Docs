\chapter*{Anotācija}
\thispagestyle{empty}

Bakalaura darba mērķis ir izstrādāt kaudzes atkļūdošanas metodi, kura ir balstīta uz atmiņās izmetes analīzi un ļauj bez tiešas piekļuves sistēmai atrast kaudzes problēmas programmā.
Pētījuma rezultātā tiek identificētas kaudzes problēmu pazīmes atmiņas izmetē un ir piedāvāta kaudzes atkļūdošanas metode.

Izstrādājamā metode ir balstīta uz GNU C bibliotēkas iedalītāju un ir nodemonstrēta, izmantojot 3 kaudzes problēmu piemērus: atmiņas noplūdi, fragmentēšanu un maksimālās atmiņas izmantošanas problēmu.
%(sākot ar versiju 2.3)

Piedāvātā metode tika pārbaudīta, izstrādājot analizatorus, katrai izvēlētai problēmai, kuri ļauj parādīt atkļūdošanas metodi darbībā un liecina par to, ka izstrādājamā metode strādā un var tikt pielietota kaudzes atkļūdošanai.
%Lai pārliecinātos kā metode strādā un var tikt pielietota kaudzes atkļūdošanai 


%Darbā ir aprakstīti jēdzieni, uz kuriem balstīta metode, dots īss ieskats atmiņas organizācijā un GNU C bibliotēkas ptmalloc2 realizācijā un ir pētītas 3 kaudzes problēmas: atmiņas noplūde, fragmentēšana, maksimālās atmiņas izmantošanas problēma.

Darbs sastāv no ievada, 4 nodaļam, secinājumiem un 5 pielikumiem. Tajā ir 50 lappuses, \totfig\ attēli, \tottab\ tabulas pamattekstā un \total{citenum} nosaukumi literatūras sarakstā.

\textbf{Atslēgvārdi:} atmiņas izmete, atkļūdošanas metode, kaudze, glibc.

\newpage

\chapter*{Abstract}
\begin{center}
\linespread{1.2}
\vspace{-0.3cm}
\large \textbf {The development of a heap debugging method \protect\\  based on the use of core dumps}
\end{center}

The purpose of this paper is to develop a heap debugging method which is based on core dump analysis and allows finding heap problems in the program without direct access to the system.
The research has identified characteristics of heap memory problems in core dumps, a method for heap debugging is proposed.

The method is based on GNU C library's implementation of heap memory allocator and is demonstrated using three problems of heap: memory leak, fragmentation, peak memory utilization.

The proposed method was verified by implementing analyzers for each of the problems.
The analyzers demonstrate that the proposed method is feasible for heap memory debugging.


\thispagestyle{empty}
	The work consists of introduction, 4 chapters, conclusions and 5 appendixes. It contains 50 pages, \totfig\ figures, \tottab\ tables and \total{citenum} references.

\textbf{Keywords:} core dump, debugging method, heap, glibc.


\newpage 






