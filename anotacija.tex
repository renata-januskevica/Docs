\chapter*{Anotācija}
\thispagestyle{empty}
	

	Darbs sastāv no ievada, 6 nodaļām, secinājumiem un 3 pielikumiem. Tajā ir \pageref{LastPage} lappuses, \totfig\ attēli, \tottab\ tabulas pamattekstā un \total{citenum} nosaukumi literatūras sarakstā.

Atslēgvārdi: atmiņas izmete, atkļūdošanas metode, kaudze, GNU C.
\newpage
\chapter*{Abstract}
\begin{center}
\linespread{1.2}
\vspace{-0.3cm}
\large \textbf {The development of a heap debugging method \protect\\  based on the use of core dumps}
\end{center}

\thispagestyle{empty}
	The work consists of introduction, 6 chapters, conclusions and 3 appendixes. It contains \pageref{LastPage} pages, \totfig\ figures, \tottab\ tables and \total{citenum} references.

Keywords: core dump, debugging method, heap, GNU C.


\newpage 