\chapter*{Anotācija}
\thispagestyle{empty}

Bakalaura darba mērķis ir izstrādāt kaudzes atkļūdošanas metodi, kura balstīta uz atmiņās izmetes analīzi un ļauj attālināti atrast problēmu lietotnē.
Pētījuma rezultātā tiek identificētas atmiņas noplūdes, maksimālās atmiņas izmantošanas problēmas, fragmentēšanas problēmas pazīmes atmiņas izmetē un ir parādīts kā, izmantojot atmiņas izmeti, ir iespējams atkļūdot kaudzi.

Darba ir aprakstīti jēdzieni, uz kuriem balstīta metode, izpētītas kaudzes problēmas un piedāvāta kaudzes atkļūdošanas metode, kura tiek pārbaudīta trijos analizatoru piemēros.
Realizētie analizatori demonstrē atkļūdošanas metodi darbībā un parāda, ka metode var tikt pielietota kaudzes problēmu atkļūdošanai.

Darbs sastāv no ievada, 4 nodaļam, secinājumiem un 5 pielikumiem. Tajā ir 50 lappuses, \totfig\ attēli, \tottab\ tabulas pamattekstā un \total{citenum} nosaukumi literatūras sarakstā.

\textbf{Atslēgvārdi:} atmiņas izmete, atkļūdošanas metode, kaudze, glibc.

\newpage

\chapter*{Abstract}
\begin{center}
\linespread{1.2}
\vspace{-0.3cm}
\large \textbf {The development of a heap debugging method \protect\\  based on the use of core dumps}
\end{center}

\thispagestyle{empty}
	The work consists of introduction, 4 chapters, conclusions and 5 appendixes. It contains 50 pages, \totfig\ figures, \tottab\ tables and \total{citenum} references.

\textbf{Keywords:} core dump, debugging method, heap, glibc.


\newpage 