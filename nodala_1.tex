Šajā nodalījumā tiek aplūkots atmiņas izmetes jēdziens, ka arī aprakstītas atmiņas izmetes ģenerēšanas iespējas un nosacījumi.
Nodalījumā ir aprakstīts atkļūdošanas process, kas var būt paveikts, izmantojot atmiņas izmeti.
Uz šiem pamatjēdzieniem, turpmāk tiks balstīta izstrādājamā kaudzes atkļūdošanas metode.
\section{Atmiņas izmete}

Sistēmās, kuras atbalsta POSIX standartus, ir signāli \cite{USP}, kuri, pēc noklusētās apstrādes, izraisa atmiņas izmetes ģenerēšanu un pārtrauc procesa darbību. 
Šos signālus var atrast  \texttt{man 7 signal} komandas izvadā. 
Signāliem, kuri izraisa izmetes ģenerēšanu, signālu tabulā \cite{signal} ir lauks ar vērtību core, kas atrodas ailē ar nosaukumu darbība (Action). 
Uzģenerētā atmiņas izmete iekļauj sevī procesa atmiņas attēlojumu uz procesa pārtraukšanas brīdi, piemēram, CPU reģistrus un steka vērtības katram pavedienam, globālos un statiskos mainīgos. 
Atmiņas izmeti var ielādēt atkļūdotājā, tāda kā gdb, lai  apskatītu programmas stāvokli uz brīdi, kad atnāca operētājsistēmas signāls \cite{core}.
Veicot atmiņas izmetes analīzi, kļūst iespējams atrast un izlabot kļūdas, pat tad, ja nav piekļuves sistēmai. 


Eksistē vairākas iespējas kā var uzģenerēt atmiņas izmeti.  To var izdarīt no programmas koda,  gdb atkļūdotāja vai komandrindas interpretatora. 
Turpmāk katra no iespējam tiks apskatīta sīkāk.

\subsection{Atmiņas izmetes ģenerēšana no koda}

Ģenerējot atmiņas izmeti no programmas koda, ir divas iespējas: process var turpināties vai beigt savu darbu pēc signāla nosūtīšanas.

\begin{figure}[h]
\begin{lstlisting}
#include <signal.h>

int main ()  {

    raise(SIGSEGV); /* Signal for Invalid memory reference */
	
    return 0;
}
\end{lstlisting}
\caption{\textbf{\fontsize{11}{12}\selectfont {Atmiņas izmetes ģenerēšana, pārtraucot procesa darbību}}}
\end{figure}

Ja nav nepieciešams, lai process turpinātu darbību, tad var izmantot funkcijas raise(), abort(), kā arī var apzināti pieļaut kļūdu kodā.
Tādas kļūdas kā dalīšana ar nulli nosūta SIGFPE signālu, bet vēršanās pēc rādītāja ar null vērtību - SIGSEGV signālu.
Izmantojot funkciju raise(), ir iespējams norādīt atmiņas izmeti izraisošo signālu.
Piemērā (sk. 1.1. attēlu) ir redzams C kods, kur funkcija raise() nosūta SIGSEGV signālu izpildāmai programmai. 
Pēc šī izsaukuma izpildes tiek izvadīts ziņojums: Segmentation fault (core dumped).
Atmiņas izmeti lietotāju procesiem var atrast darba mapē, jo Linux operētājsistēmā tā ir noklusēta atmiņas izmetes atrašanas vieta, bet noklusētais atmiņas izmetes nosaukums ir core.

Ir iespējams uzģenerēt atmiņas izmeti, nepārtraucot procesa darbību (sk. 1.2. attēlu). 
To var panākt ar fork() funkcijas palīdzību. Funkcija fork() izveido bērna procesu, kas ir vecāka procesa kopija.
Funkcijas fork() veiksmīgas izpildes gadījumā, bērnu procesam atgriež 0 vērtību. 
Pēc abort() funkcijas izpildes, bērns beidz izpildi un uzģenerē atmiņas izmeti. Vecāks process turpina izpildi.


\begin{figure}[h]
\begin{lstlisting}
 #include <stdlib.h>

 int main () {
 
     int child = fork();
     if (child == 0) {
         abort(); /* Child */
     }
     return 0;
 }
 \end{lstlisting}
\caption{\textbf{\fontsize{11}{12}\selectfont {Atmiņas izmetes ģenerēšana, turpinot procesa darbību}}}
\end{figure}

\subsection{Atmiņas izmetes ģenerēšana no gdb}
Izsaukumi no koda nav vienīga iespēja kā varētu iegūt atmiņas izmeti. 
Var izmantot gdb komandas: \texttt{generate-core-file [file]}(sk. 1.3. attēlu) vai \texttt{gcore [file]}. Šīs komandas izveido gdb pakļautā procesa atmiņas izmeti. 
Izmantojot gdb, var uzģenerēt atmiņas izmeti, kura atbilst kādam pārtraukuma punkta stāvoklim. 
Neobligāts arguments \texttt{filename} nosaka atmiņas izmetes nosaukumu. Šī gdb komanda ir realizēta GNU/Linux, FreeBSD, Solaris and S390 sistēmās \cite {gdb_gen_core}.

\begin{figure}[h]
\begin{lstlisting}
(gdb) attach <pid>
(gdb) generate-core-file <filename>
(gdb) detach
(gdb) quit
 \end{lstlisting}
\caption{\textbf{\fontsize{11}{12}\selectfont {Atmiņas izmetes ģenerēšana, izmantojot gdb}}}
\end{figure}

\subsection{Atmiņas izmetes ģenerēšana no komandrindas interpretatora}
Trešā iespēja ir nosūtīt signālu, izmantojot komandrindas interpretatoru. 
Komanda kill var nosūtīt jebkuru signālu procesam.
Pēc komandas  \texttt{kill -<SIGNAL\_NUMBER> <PID>}, signāls ar numuru  SIGNAL\_NUMBER tiks nosūtīts procesam ar norādītu PID vērtību.
Izmantojot shell komandrindas interpretatoru ir  iespējams izmantot īsinājumtaustiņus signālu nosūtīšanai. 
Nospiežot Control + $\backslash$ tiks nosūtīts SIGQUIT signāls procesam, kas pašreiz ir palaists (sk. 1.4. attēlu) \cite {nosacijumi}. 
Šajā piemēra ziņojumu - Quit (core dumped), izdruka shell. 
Šīs komandrindas interpretators noteic, ka  sleep procesu (shell bērnu) pārtrauca SIGQUIT signāls. 
Pēc šī signāla nosūtīšanās darba mapē tiek uzģenerēta atmiņas izmete. 

\begin{figure}[h]
\begin{lstlisting}
$ ulimit –c unlimited
$ sleep 30
Type Control +\
^\Quit (core dumped)
\end{lstlisting}
\caption{\textbf{\fontsize{11}{12}\selectfont {Atmiņas izmetes ģenerēšana, izmantojot īsinājumtaustiņus}}}
\end{figure}

\subsection{Atmiņas izmetes ģenerēšanas nosacījumi}
Lai  uzģenerētu atmiņas izmeti ir jābūt izpildītiem sekojošiem nosacījumiem \cite {nosacijumi}:
%\vspace{-6pt}
\begin{itemize}
	\item   ir jānodrošina atļauja procesam rakstīt core datni darba mapē,
	\item 	ja datne, ar vienādu nosaukumu jau eksistē, tad ir jābūt ne vairāk kā vienai stingrai saitei, 
	\item 	izvēlētai darba mapei ir jābūt reālai un jāatrodas norādītajā vietā, 
	\item 	Linux core datnes izmēra robežai {RLIMIT\_CORE} jāpārsniedz ģenerējamā faila izmēru, { RLIMIT\_FSIZE} robežai jāļauj procesam izveidot atmiņas izmeti,
	\item 	ir  jāatļauj lasīt bināro datni, kura ir palaista,
	\item 	failu sistēmai, kurā atrodas darba mape, ir jābūt uzmontētai priekš rakstīšanas, tai nav jābūt pilnai un ir jāsatur brīvie indeksa deskriptori,
	\item 	bināro datni jāizpilda lietotājam, kurš ir datnes īpašnieks (group owner).
\end{itemize} 
 
 Pēc noklusējuma atmiņas izmetes ģenerēšanas iespēja ir izslēgta,  \texttt{ulimit -c unlimited} komanda ļauj ieslēgt atmiņas izmetes ģenerēšanu.

\section{Atkļūdošana, izmantojot atmiņas izmeti}
Atmiņas izmete satur datus, kuri dod iespēju atrast kļūdas. Tāpēc atmiņas izmete var tikt pielietota, lai veiktu lietotnes atkļūdošanu, pēc neparedzētas programmas apstāšanās.
Atmiņas izmetes analīze ir efektīvs veids, kā var attālināti atrast un izlabot kļūdas bez iejaukšanās un tiešas piekļuves sistēmai.
Daudzos gadījumos, atmiņas izmete ir speciāli uzģenerēta datne, kura palīdz iegūt atmiņas stāvokli uz signāla nosūtīšanas brīdi.
Atmiņas izmete ir labi piemērota kļūdu meklēšanai, kas saistītas ar nepareizo atmiņas izmantošanu lietotnē.

Atmiņas izmete ir ELF formāta datne. 
ELF formāts ir Linux un Unix standarts priekš izpildāmām datnēm, objektu datnēm, bibliotēkām un atmiņas izmetēm.
Lai darbotos ar atmiņas izmetēm ir nepieciešams, lai rīks, kurš tika izvēlēts (bibliotēka, utilītprogramma vai atkļūdotājs) atbalstītu ELF formāta datnes.
GNU gdb ir  Linux standarta atkļūdotājs \cite{MWMK}, kurš ir plaši pielietojams atmiņas izmešu analīzei. 
Turpmāk tiek apskatīta atmiņas izmetes analīze ar gdb atkļūdotāja piemēru.

\subsubsection{Atmiņas izmetes atkļūdošana, izmantojot gdb }
Ja atmiņas izmetes analīzei tika izvēlēts GNU gdb atkļūdotājs, tad pirms sākt analīzi ir nepieciešams pārliecināties ka gdb ir pareizi nokonfigurēts priekš procesora arhitektūras, no kuras bija iegūta atmiņas izmete.
To var identificēt uzreiz pēc gdb palaišanas, ar sekojošās rindiņas palīdzību: \texttt{This GDB was configured as i686-linux-gnu}.  
Lai atmiņas izmete saturētu atkļūdošanas informāciju, ir jānorāda -g opcija kompilācijas laikā.
Atkļūdošanas informācija ir uzglabāta objektu datnē un saglabā atbilstību starp izpildāmo datni un pirmkodu, ka arī mainīgo un funkciju datu tipus.
Ja atmiņas izmete neiekļauj atkļūdošanas informāciju, tad atmiņas izmete var attēlot sekojošo tekstu (sk. 1.5. attēlu).

\begin{figure}[h]
\begin{lstlisting}
(gdb) p main
$ 1 = {<text variable, no debug info>} 0x80483e4 <main>
\end{lstlisting}
\caption{\textbf{\fontsize{11}{12}\selectfont {Atmiņas izmete nesatur atkļūdošanas informāciju}}}
\end{figure}

Kad atmiņas izmete ir uzģenerēta, tad to var apskatīt, izmantojot gdb atkļūdotāju (sk. 1.6. attēlu). 
Atkļūdotājam kā argumenti tiek padoti: izpildes fails un atmiņas izmete. 
Izpildes failam ir jāatbilst atmiņas izmetei, lai varētu apskatīt korektus, nesabojātus datus.

\begin{figure}[h]
\begin{lstlisting}
$ gdb <path/to/the/binary> <path/to/the/core>
\end{lstlisting}
\caption{\textbf{\fontsize{11}{12}\selectfont {Atmiņas izmetes atvēršana, izmantojot gdb atkļūdotāju}}}
\end{figure}
%$

Gdb ļauj iegūt svarīgus datus no atmiņas izmetes. Komanda \texttt{info files} ļauj apskatīt procesa segmentus. 
Katram segmentam ir adrešu apgabals ar nosaukumu. 
Segmenti, kuru nosaukums ir "loadNNN" pieder procesam, tajos var tikt uzglabāti: statiskie dati, steks, kaudze, koplietošanas atmiņa.
Tā kā segmentu robežas ir zināmas, tad kļūst iespējams izdrukāt atmiņas saturu, kas pieder segmentiem, uzzināt kuram segmentam pieder konkrētā atmiņas adrese un kādas ir segmentu adrešu vērtības.

Lai izdrukātu atmiņas apgabalu ir nepieciešams norādīt atmiņas adresi (\texttt{addr}), no kura sākt atmiņas izdruku, formātu (\texttt{f}), apgabala lielumu (\texttt{n}) un norādīt vienības lielumu (\texttt{u}) (sk. 1.7. attēlu). 
Izmantojot doto piemēru tiks izdrukāts \texttt{n} liels atmiņas apgabals, kurš sākas ar adresi \texttt{addr}. Formātu un vienības lielumu vajag norādīt saskaņā ar gdb pamācību \cite{gdb}. 

\begin{figure}[h]
\begin{lstlisting}
(gdb) x/nfu addr
\end{lstlisting}
\caption{\textbf{\fontsize{11}{12}\selectfont {Atmiņas apgabala izdrukāšanas formāts}}}
\end{figure}

Atmiņas izmetes analīze sākas ar backtrace izdrukāšanu. 
Bactrace ir pārskats, kurš attēlo kā programma nonāca stāvoklī, kurā pabeidza savu darbību.
Tas palīdz ātri atrast instrukciju kura bija izpildīta pēdēja un daudzos gadījumos, tas palīdz ātri identificēt kļūdas cēloni.
Katra rindiņa satur rāmi (frame). Bactrace izdruka sākas ar rāmi, kurš iekļauj funkciju, kura bija izpildīta pēdēja. 
Nākamais rāmis iekļauj funkciju, kas izsauca iepriekšējā rāmī iekļauto funkciju.
Katrai baktrace rindiņai piešķirts rāmja numurs. Katrs rāmis var iekļaut: funkcijas nosaukumu, pirmkoda datnes nosaukumu, pirmkodam atbilstošo rindiņas numuru un funkcijas argumentus. 
Bactrace var tikt iegūts izmantojot gdb komandu \texttt{bactrace full} vai \texttt{bt f}. 
Pēc noklusējuma, daudzpavedienu lietotnēs gdb  rāda bactrace kārtējām pavedienam, bet pastāv iespēja iegūt arī bactrace izdruku priekš citiem pavedieniem.
Ja programma bija nokompilēta ar optimizācijas opciju, tad bactrace varētu neiekļaut funkcijas argumentus, tad funkciju argumenti varētu tikt nodoti caur CPU reģistriem.
CPU reģistru vērtības ir iespējams iegūt, izmantojot komandu \texttt{info registers} vai \texttt{i r}.
Atmiņas izmetē atrodas pēdējais atmiņas stāvoklis, tāpēc CPU reģistru vērtības ir iespējams atjaunot no steka, ja pēc izjaukšanas (disassembling) ir redzams, ar cik lielu nobīdi tie tika saglabāti stekā.
Izjaukšana (disassembling) ļauj izdrukāt asamblera instrukcijas noraidītai funkcijai. 
Tas dod iespēju salīdzināt pirmkodu ar asamblera instrukcijām un tāda veidā var atrast nepieciešamo mainīgo vērtības stekā.

