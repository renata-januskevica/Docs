%Šajā nodaļā tiek aplūkoti atmiņas izmetes un balstītās uz tās atkļūdošanas metodes pamatjēdzieni, ka arī aprakstītas atmiņas izmetes ģenerēšanas iespējas un nosacījumi.

\section{Atmiņas izmete}

Sistēmās, kuras atbalsta POSIX standartus, ir signāli \cite {USP}, kuri, pēc noklusētās apstrādes, izraisa atmiņas izmetes ģenerēšanu un pārtrauc procesa darbību. 
Šos signālus var atrast  \texttt{man 7 signal} komandas izvadā. 
Signāliem, kuri izraisa izmetes ģenerēšanu, signālu tabulā \cite{signal} ir lauks ar vērtību core, kas atrodas ailē ar nosaukumu darbība (Action). 
Uzģenerētā atmiņas izmete iekļauj sevī procesa atmiņas attēlojumu uz procesa pārtraukšanas brīdi, piemēram, CPU reģistrus un steka vērtības katram pavedienam, globālos un statiskos mainīgos. 
Atmiņas izmeti var ielādēt atkļūdotājā, tāda kā gdb, lai  apskatītu programmas stāvokli uz brīdi, kad atnāca operētājsistēmas signāls \cite {core}.
Veicot atmiņas izmetes analīzi, kļūst iespējams atrast un izlabot kļūdas, pat tad, ja nav piekļuves sistēmai. 


Eksistē vairākas iespējas kā var uzģenerēt atmiņas izmeti.  To var izdarīt no programmas koda,  gdb atkļūdotāja vai komandrindas interpretatora. 
Turpmāk katra no iespējam tiks apskatīta sīkāk.

\subsection{Atmiņas izmetes ģenerēšana no koda}

Ģenerējot atmiņas izmeti no programmas koda, ir divas iespējas: process var turpināties vai beigt savu darbu pēc signāla nosūtīšanas.

\begin{figure}[h]
\begin{lstlisting}
#include <signal.h>

int main ()  
{
    raise(SIGSEGV); /* Signal for Invalid memory reference */
	
    return 0;
}
\end{lstlisting}
\caption{\textbf{\fontsize{11}{12}\selectfont {Atmiņas izmetes ģenerēšana, pārtraucot procesa darbību}}}
\end{figure}

Ja nav nepieciešams, lai process turpinātu darbību, tad var izmantot funkcijas raise(), abort(), kā arī var apzināti pieļaut kļūdu kodā.
Tādas kļūdas kā dalīšana ar nulli nosūta SIGFPE signālu, bet vēršanās pēc rādītāja ar null vērtību - SIGSEGV signālu.
Izmantojot funkciju raise(), ir iespējams norādīt atmiņas izmeti izraisošo signālu.
Piemērā (sk. 1.1. attēls) ir redzams C kods, kur funkcija raise() nosūta SIGSEGV signālu izpildāmai programmai. 
Pēc šī izsaukuma izpildes tiek izvadīts ziņojums: Segmentation fault (core dumped).
Atmiņas izmeti var atrast darba mapē, jo Linux operētājsistēmā tā ir noklusēta atmiņas izmetes atrašanas vieta, bet noklusētais atmiņas izmetes nosaukums ir core.

Ir iespējams uzģenerēt atmiņas izmeti, nepārtraucot procesa darbību (sk. 1.2. attēls). 
To var panākt ar fork() funkcijas palīdzību. Funkcija fork() izveido bērna procesu, kas ir vecāka procesa kopija.
Funkcijas fork() veiksmīgas izpildes gadījumā, bērnu procesam atgriež 0 vērtību. 
Pēc abort() funkcijas izpildes, bērns beidz izpildi un uzģenerē atmiņas izmeti. Vecāks process turpina izpildi.


\begin{figure}[h]
\begin{lstlisting}
 #include <stdlib.h>

 int main ()
 {
     int child = fork();
     if (child == 0) {
         abort(); /* Child */
     }
     return 0;
 }
 \end{lstlisting}
\caption{\textbf{\fontsize{11}{12}\selectfont {Atmiņas izmetes ģenerēšana, turpinot procesa darbību}}}
\end{figure}

\subsection{Atmiņas izmetes ģenerēšana no gdb}
Izsaukumi no koda nav vienīga iespēja kā varētu iegūt atmiņas izmeti. 
Var izmantot gdb komandas: \texttt{generate-core-file [file]}(sk. 1.3. attēls) vai \texttt{gcore [file]}. Šīs komandas izveido gdb pakļautā procesa atmiņas izmeti. 
Izmantojot gdb, var uzģenerēt atmiņas izmeti, kura atbilst kādam pārtraukuma punkta stāvoklim. 
Neobligāts arguments \texttt{filename} nosaka atmiņas izmetes nosaukumu. Šī gdb komanda ir realizēta GNU/Linux, FreeBSD, Solaris and S390 sistēmās \cite {gdb_gen_core}.

\begin{figure}[h]
\begin{lstlisting}
(gdb) attach <pid>
(gdb) generate-core-file <filename>
(gdb) detach
(gdb) quit
 \end{lstlisting}
\caption{\textbf{\fontsize{11}{12}\selectfont {Atmiņas izmetes ģenerēšana, izmantojot gdb}}}
\end{figure}

\subsection{Atmiņas izmetes ģenerēšana no komandrindas interpretatora}
Trešā iespēja ir nosūtīt signālu, izmantojot komandrindas interpretatoru. 
Komanda kill var nosūtīt jebkuru signālu procesam.
Pēc komandas  \texttt{kill -<SIGNAL\_NUMBER> <PID>}, signāls ar numuru  SIGNAL\_NUMBER tiks nosūtīts procesam ar norādītu PID vērtību.
Izmantojot shell komandrindas interpretatoru ir  iespējams izmantot signālu īsinājumtaustiņus. 
Nospiežot Control + $\backslash$ tiks nosūtīts SIGQUIT signāls procesam, kas pašreiz ir palaists (sk. 1.4. attēls). 
Šajā piemēra ziņojumu - Quit (core dumped), izdruka shell. 
Šīs komandrindas interpretators noteic, ka  sleep procesu (shell bērnu) pārtrauca SIGQUIT signāls. 
Pēc šī signāla nosūtīšanās darba mapē tiek uzģenerēta atmiņas izmete. 

\begin{figure}[h]
\begin{lstlisting}
$ ulimit –c unlimited
$ sleep 30
Type Control +\
^\Quit (core dumped)
\end{lstlisting}
\caption{\textbf{\fontsize{11}{12}\selectfont {Atmiņas izmetes ģenerēšana, izmantojot īsinājumtaustiņus}}}
\end{figure}

\subsection{Atmiņas izmetes ģenerēšanas nosacījumi}
Lai  uzģenerētu atmiņas izmeti ir jābūt izpildītiem sekojošiem nosacījumiem \cite {nosacijumi}:
%\vspace{-6pt}
\begin{itemize}
	\item   ir jānodrošina atļauja procesam rakstīt core datni darba mapē,
	\item 	ja datne, ar vienādu nosaukumu jau eksistē, tad ir jābūt ne vairāk kā vienai stingrai saitei, 
	\item 	izvēlētai darba mapei ir jābūt reālai un jāatrodas norādītajā vietā, 
	\item 	Linux core datnes izmēra robežai {RLIMIT\_CORE} jāpārsniedz ģenerējamā faila izmēru, { RLIMIT\_FSIZE} robežai jāļauj procesam izveidot atmiņas izmeti,
	\item 	ir  jāatļauj lasīt bināro datni, kura ir palaista,
	\item 	failu sistēmai, kurā atrodas darba mape, ir jābūt uzmontētai priekš rakstīšanas, tai nav jābūt pilnai un ir jāsatur brīvie indeksa deskriptori,
	\item 	bināro datni jāizpilda lietotājam, kurš ir datnes īpašnieks (group owner).
\end{itemize} 



\section{Atkļūdošana, izmantojot atmiņas izmeti}
Core datne var būt izmantota, lai veiktu lietotnes atkļūdošanu. 


\subsection{Simboli un atkļūdošanas informācija}

Struktūras un objekti ir tikai abstrakcijas, kuras tiek interpretētas kā nobīdes atmiņā. Kompilators pārveido austa līmeņa kodu mašīnkodā. 
Kopā ar mašīnkodu kompilators var uzģenerēt atkļūdošanas informāciju, kura attēlo saistību starp izpildāmo programmu un oriģinālo pirmkodu.
Atkļūdošanas informācija tiek iekodēta kāda iepriekš-definētā formātā. Viens no šādiem formātiem ir DWARF, kurš tiek izmantots ELF datnēs.
DWARF ir komplicēts formāts, kurš bija veidots priekš dažādām arhitektūram un operētājsistēmām.

ELF formāts satur patvaļīgas sadaļas, kuras var būt iekļautas jebkurā objektu datnē. Sadaļu tabulā atrodas visu eksistējošo sadaļu nosaukumi.
Komanda \texttt{objdump -h} ļauj apskatīt sadaļas ar nosaukumiem. Nosaukumi, kuri sākas ar \texttt{.debug\_} ir DWARF atkļūdošanas sadaļas.
Dažādi rīki var apstrādāt ELF datnes.



