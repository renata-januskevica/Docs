\section{Analizatora darbības princips}

 \begin{figure}[h]
\begin{center}
\includegraphics[scale=0.6]{method}
\end{center}
\caption{\textbf{\fontsize{11}{12}\selectfont {Uz metodi balstītā algoritma blok-shēma}}}
\end{figure}

Blok-shēmā (sk. 4.1. attēlu) ir attēlots kaudzes atkļūdošanas algoritms.
Šī algoritma ievaddati ir izpildāmā datne un atmiņas izmete.
Atmiņas izmete var būt bojāta, tāpēc pirms sākt atkļūdošanas procedūru ir nepieciešams veikt atmiņas izmetes validēšanu (1).
Šeit ir jāveic dažas pārbaudes, piemēram, pārbaudi uz neatbilstību formātam vai atmiņas izmetes nogriezto saturu.
Kā pārbaudīt vai izmetei ir nogriezts saturs ir aprakstīts \ref{sec:validaty} sadaļā.
Validēšana palīdz savlaicīgi atklāt, ka atmiņas izmete nav derīga analīzei un pārtraukt algoritma darbību.
Nākamā pārbaude ir veikta pēc izpildāmās datnes nolasīšanas.
Ir iespējams, ka izpildāmā datne neatbilstis atmiņas izmetei (2), tas ir, kad tiek izmantota cita izpildāmā datne atmiņas izmetes ģenerēšanai vai atmiņas izmetei neatbilst izpildāmās datnes versija.
Ir nepieciešams pārbaudīt atbilstību un jāpārliecinās, ka var piekļūt galvenās arēnas datiem.
Pēc dotām pārbaudēm var sākt atmiņas izmetes analīzi.
Blok-shēmā ir parādīts, ka ir nepieciešams darbināt analizatorus, kuri  pārbauda problēmu pazīmes atmiņas izmetē.
Daudzkodolu procesoriem ir iespējams realizēt algoritmu, kur analizatori strādās paralēli, vienkodolā visi analizatori izpildīsies secīgi.
Analizatoru skaits nav ierobežots, taču bakalaura darba algoritms tiks nodemonstrēts ar 3 analizatoru piemēriem (3. atmiņas noplūdes analizators, 4. maksimālās atmiņas izmantošanas problēmas analizators, 5. fragmentēšanas problēmas analizators).
Katrs analizators pārbauda noteiktās kaudzes problēmas pazīmes.
Pēc tam tiek izvadīts kopējais rezultāts (6).
Ja ir zināma programmai raksturīga uzvedība, tad pēc rezultāta izvadīšanas var secināt pār problēmām.

Pieeja tiek izstrādāta ievērojot sekojošus ierobežojumus: %, kas aprakstīta bakalaura darbā,
\begin{itemize}
	\item GNU C bibliotēkas izmantošana sākot ar 2.3 versiju;
    \item ELF atmiņas izmetes formāts;
    \item analizātori nedod pilnīgu secinājumu par problēmas esamību, bet sniedz informāciju par sistēmas stāvokli, kura ļauj izstradatājam secināt par problēmu.
\end{itemize}



\section{Atmiņas noplūdes analizators}

\section{Maksimālās atmiņas izmantošanas problēmas analizators}
Tā kā maksimālās atmiņas izmantošanas problēma ir tuva fragmentēšanai, tad autore nolēma realizēt divus analizatorus vienā gdb skriptā.
Šajā sadaļā tiek aprakstīts maksimālās atmiņas izmantošanas problēmas analizators (sk. 4. pielikumu).
Skriptā, kurš realizē divu problēmu analizatorus, ir nodefinēta komanda \texttt{analyze}, kura izsauc parējās komandas priekš abiem analizatoriem.
Šai komandai no gdb atkļūdotāja ir nepieciešams padot divus argumentus: galvenās arēnas adresi un skaitli, kurš norāda cik sīki būtu jāsadala kaudze.

\begin{figure}[h]
\begin{lstlisting} [style=customgdb]
------------- Maksimālā atmiņas izmantošanas problēma --------
Atbrīvoto un iedalīto gabalu attiecībā:
Apgabals 0x845e000 - 0x85c32f4 7%
Apgabals 0x85d5210 - 0x873a504 0%
Apgabals 0x874c288 - 0x88b157c 0%
Apgabals 0x88c3300 - 0x8a285f4 0%
Apgabals 0x8a3a378 - 0x8b9f66c 0%
Apgabals 0x8bb13f0 - 0x8d166e4 0%
Apgabals 0x8d28468 - 0x8e8d75c 0%
\end{lstlisting}
\caption{\textbf{\fontsize{11}{12}\selectfont {Maksimālā atmiņas izmantošanas rādītājs}}}
\end{figure}


Atkarībā no otrā argumenta kaudze tiks sadalīta vienā vai vairākos atsevišķos apgabalos.
Katram kaudzes apgabalam tiks izrēķināta attiecība: kopējais atbrīvoto gabalu izmērs apgabalā pret kopējo atbrīvoto un iedalīto gabalu izmēru apgabalā.
Šī attiecība ļauj iegūt daļu, kuru aizņem atbrīvotie gabali no kopējā kaudzes apgabala satura (sk. 4.2. attēlu).
Zinot to, ka maksimālās atmiņas izmantošanas problēmas laikā, atbrīvotie gabali nav vienmērīgi izkliedēti kaudzē, bet atrodas blakus top gabalam, kļūst iespējams atrast problēmu pēc šī radītāja.
Jā kaudzēs apgabalā, kurš atrodas blakus top apgabalam rādītājs ir lielāks par pārējiem apgabaliem, tad tās var liecināt par maksimālās atmiņas izmantošanas problēmu.
Blakus atbrīvoto gabalu daļai ir norādīta atmiņas gabala sākuma un beigu adrese.
Pēc šīm adresēm var uzzināt atmiņas daudzumu, kurš netika atgriezts operētājsistēmai.

Algoritma realizācijai bija nepieciešams apstaigāt kaudzi un sakrāt datus par katru kaudzes apgabalu.
Sākumā tiek iegūta norāde uz kaudzi. 
Lai to iegūtu, ir nepieciešams iegūt top gabala adresi, jo top gabals atrodas kaudzes beigās.
Top gabals veido kaudzi, bet atmiņa, kura uzglabāta top gabalā, netiek izmantota programmā.
No top adreses vajag atņemt kaudzes izmēru, jo kaudze aug no mazākas adreses uz lielāko, un pieskaitīt top gabala izmēru.
Rezultātā tiek iegūta norāde uz kaudzes sākumu.
Sākot ar šo norādi tiek apstaigāti visi atmiņas gabali kaudzē.



\section{Fragmentēšanas analizators}
Šajā sadaļā tiek aprakstīts fragmentēšanas analizators, kurš tika realizēts gdb skriptā (sk. 4. pielikumu).
Fragmentēšanas problēmas novērtēšanai analizators izvada 2 svarīgākus rādītājus (sk. 4.3. attēlu): lielāko un kopējo atbrīvoto gabalu izmēru arēnā.
Problēma būs novērojama, kad tiks iegūts tāds maksimālais atbrīvotais gabals, kurš nevar apmierināt pieprasījumu pēc atmiņas, ja kopējais izmērs ir pietiekošs.

\begin{figure}[h]
\begin{lstlisting} [style=customgdb]
Bin numurs 1: 50 gabals(-li) (8196 - 8196 baiti), kopumā = 409800 baiti

------------------------ Fragmentēšana -----------------------
Lielākā gabala izmērs: 8196 baiti (8 KiB, 0 MiB),
Kopējā atbrīvotā atmiņa bin sarakstos: 409800 baiti (400 KiB, 0 MiB),
\end{lstlisting}
\caption{\textbf{\fontsize{11}{12}\selectfont {Fragmentēšanas rādītāji}}}
\end{figure}

Lai iegūtu abus šos rādītājus ir nepieciešams apstaigāt 128 bin sarakstus un no ikviena saraksta iegūt katra atmiņas gabala izmēru.
Parastie bin saraksti atrodas galvenajā arēnā, tāpēc piekļūt tiem var, ja ir zināma arēnas struktūra un tās sākuma adrese.
Gdb skriptā katram sarakstam ir numurs no 0 līdz 127 un apstaigāšana notiek, izmantojot nobīdes no galvenās arēnas sākuma un ņemot vērā kārtējā saraksta numuru.
Katrs gabals norāda uz nākamo un iepriekšējo atmiņas gabalu, tāpēc visus saraksta gabalus var apstaigāt, pārvietojoties pa sarakstu.
Pēdējais atmiņas gabals norāda uz kārtēja saraksta sākumu.
Saraksta apstaigāšana ir jābeidz, kad ir iegūta apstrādājamā saraksta sākuma adrese.
Katram gabalam tiek pārbaudīts vai tekošais gabala izmērs nav lielāks par maksimālo gabala izmēru sarakstā un tiek atjaunināta kopējā saraksta izmēru summa.
Salīdzinot iegūtās vērtības sarakstiem tiek noteikti dotie rādītāji galvenajai arēnai.
Gabalu apstaigāšana sarakstā ir nodrošināta ar free\_chunk\_list komandu.

Pirms izvadīt lielāko un kopējo atbrīvoto gabalu izmēru rādītājus, tiek izdrukāta statistika par katru no bin sarakstiem.
Tas palīdz iegūt detalizētu statistiku par visiem parastajiem sarakstiem, kuri nav tukši.
Tas ir: skaits cik ir atbrīvoto gabalu sarakstā, amplitūda (mazākais gabals, lielākais gabals), kopējais gabalu izmērs sarakstā.


\section{Secinājumi}
\begin{enumerate}
\item
\end{enumerate}


