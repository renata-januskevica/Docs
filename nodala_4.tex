Šajā nodaļā ir aprakstīta metode, kura varētu tikt pielietota kaudzes atkļūdošanai. 
Šeit ir aplūkots metodes algoritms un trīs analizatoru realizācijas: atmiņas noplūdes analizators, maksimālās atmiņas izmantošanas problēmas analizators, fragmentēšanas problēmas analizators.
Realizētie analizatori ir uzskatāms demonstrējums, ka metode strādā un var tikt pielietota.
\section{Analizatora darbības princips}

 \begin{figure}[h]
\begin{center}
\includegraphics[scale=0.6]{method}
\end{center}
\caption{\textbf{\fontsize{11}{12}\selectfont {Uz metodi balstītā algoritma blok-shēma}}}
\end{figure}

Blok-shēmā (sk. 4.1. attēlu) ir attēlots kaudzes atkļūdošanas algoritms.
Šī algoritma ievaddati ir izpildāmā datne un atmiņas izmete.
Atmiņas izmete var būt bojāta, tāpēc pirms sākt atkļūdošanas procedūru ir nepieciešams veikt atmiņas izmetes validēšanu (1).
Šeit ir jāveic dažas pārbaudes, piemēram, pārbaudi uz neatbilstību formātam vai atmiņas izmetes nogriezto saturu.
To, kā var pārbaudīt vai izmetei ir nogriezts saturs ir aprakstīts \ref{sec:validaty} sadaļā.
Nākamā pārbaude ir veikta pēc izpildāmās datnes nolasīšanas.
Ir iespējams, ka izpildāmā datne neatbilstīs atmiņas izmetei (2), tas var notikt divos gadījumos.
Pirmkārt, kad tiek izmantota cita izpildāmā datne atmiņas izmetes ģenerēšanai.
Otrkārt, kad atmiņas izmetei neatbilst izpildāmās datnes versija.
Ir nepieciešams pārbaudīt atbilstību un jāpārliecinās, ka var piekļūt galvenās arēnas datiem.
Visās iepriekš aprakstītās pārbaudes palīdz savlaicīgi uzzināt, ka atmiņas izmete nav derīga analīzei un pārtraukt algoritma darbību.
Pēc dotām pārbaudēm var sākt atmiņas izmetes analīzi.
Blok-shēmā ir parādīts, ka ir nepieciešams darbināt analizatorus, kuri  pārbauda problēmu pazīmes atmiņas izmetē.
Daudzkodolu procesoriem ir iespējams realizēt algoritmu, kur analizatori strādās paralēli, vienkodolā visi analizatori izpildīsies secīgi.
Analizatoru skaits nav ierobežots, taču bakalaura darbā algoritms tiks nodemonstrēts trijos analizatoru piemēros (3. atmiņas noplūdes analizators, 4. maksimālās atmiņas izmantošanas problēmas analizators, 5. fragmentēšanas problēmas analizators).
Katrs analizators pārbauda noteiktās kaudzes problēmas pazīmes.
Pēc tam tiek izvadīts kopējais rezultāts (6).
Ja ir zināma programmai raksturīga uzvedība, tad pēc rezultāta izvadīšanas var secināt pār problēmām.

Pieeja tiek izstrādāta, ievērojot šādus ierobežojumus: %, kas aprakstīta bakalaura darbā,
\begin{itemize}
	\item GNU C bibliotēkas izmantošana sākot ar 2.3 versiju;
    \item ELF atmiņas izmetes formāts;
    \item analizātori nedod pilnīgu secinājumu par problēmas esamību, bet sniedz informāciju par sistēmas stāvokli, kura ļauj izstradatājam secināt par problēmu.
\end{itemize}



\section{Atmiņas noplūdes analizators}
Autore realizēja atmiņas noplūdes analizatoru gdb skriptā (sk. 4. pielikumu). Skriptā ir nodefinēta komanda \texttt{analyze}, kura izsauc parējās komandas.
Šai komandai no gdb atkļūdotāja ir nepieciešams padot vienu argumentu: galvenās arēnas adresi.
Atmiņas izmetei ir jābūt ar nosaukumu core un ir jāatrodas darba mapē.
Skripts darbosies uz 32 bitu datoru arhitektūras, kurā vismazāk nozīmīgie baiti tiek uzglabāti sākumā (little endian).
Skriptu ir iespējams pielāgot arī 64 bitu arhitektūrai un arhitektūrām ar visvairāk nozīmīgākiem baitiem sākumā (big endian).
Analizators ir izstrādāts vienpavedienu lietotnei ar vienu galveno arēnu.

Skripts atrod visus atmiņas gabalus kaudzē. 
Katram atmiņas gabalam, tiek pārbaudīts vai tās tiek iedalīts programmai vai atbrīvots.
Ja gabals ir atbrīvots, tad to nevajag apstrādāt.
Tātad, var nobīdīt norādi uz nākamo atmiņas gabalu līdz tiks sasniegts top gabals.
Katram iedalītām gabalam kaudzē tiek iegūta adrese, uz kuru būtu jānorāda norādei no procesa adrešu telpas.
Ja norādes nav, tad programma izdrukas pazaudēto adresi.
Pirms meklēt atmiņas adreses procesa adrešu telpā ir nepieciešams sagatavot datni ar atmiņas izmetes heksadecimālo saturu.
Šīm nolūkam ir izmantota \texttt{od -t x} komanda, jo, atšķirībā no \texttt{xxd}, komanda ļauj bez baitu apgriešanas Intel x86 arhitektūrā iegūt korektas adreses ar  vismazāk nozīmīgākiem baitiem sākumā \cite{DPT}.
%Realizējot šāda veida analizatoru, ir jāparedz gadījums, ka dati var netikt izlīdzināti 4 baitu robežai.
%Skriptā, šīm nolūkam visas adreses tika saplūdinātas.
Meklēšana notiek ar grep utilītprogrammas palīdzību.
Programmas izvads satur pazaudēto norāžu skaitu un pazaudēto atmiņas gabalu adreses (sk. 4.2. attēlu).

\begin{figure}[h]
\begin{lstlisting} [style=customgdb]
---Type <return> to continue, or q <return> to quit---
Nav atrasta norāde uz adresi: 9c63498
Nav atrasta norāde uz adresi: 9c7c4a0
Nav atrasta norāde uz adresi: 9c954a8
Nav atrasta norāde uz adresi: 9cae4b0
Nav atrasta norāde uz adresi: 9cc74b8
Procesa adrešu telpā tiek pazaudēts(i): 100 gabals(i).
\end{lstlisting}
\caption{\textbf{\fontsize{11}{12}\selectfont {Atmiņas noplūdes atrašana, gdb skripta izvads}}}
\end{figure} %$

Skripts var kļūdaini atrast pazaudētas adreses, ja gabaliem nodrošināta piekļuve ar nobīdi.
Tas notiek, kad tiek iedalīts gabals, bet no programmas nav norādes uz gabala sākumu.
Piemērā ir redzams, ka str norādei ir piešķirta atmiņas gabala adrese ar nobīdi (sk. 4.3. attēlu).
Tas ir viens no speciāliem gadījumiem, kurš netiks apstrādāts skriptā. 
Turklāt ir iespējams iegūt adresi un apskatīties datus, kuri tiek uzglabāti atmiņā.
\begin{figure}[h]
\begin{lstlisting} [language=C++]
char * str = (char *)malloc(sizeof(char) * num_elements) + 16; /* C */
\end{lstlisting}
\caption{\textbf{\fontsize{11}{12}\selectfont {Speciālgadījums, no procesa adrešu telpā nav norādes uz gabala sākumu}}}
\end{figure}

Atmiņas noplūdes analizators sameklē pazaudētus gabalus un ļauj atrast problēmu atmiņas izmetē.
Parauga programmai (sk. 1. pielikumu) ar diviem argumentiem: 100, 100 skripts izvadīja, ka tiek pazaudēti 100 atmiņas gabali.
Tā kā sniegtā izdruka atbilst programmas stāvoklim,tad tiek uzskatīts, ka dotās problēmas atkļūdošanai var izmantot atmiņas izmeti.

\section{Maksimālās atmiņas izmantošanas problēmas analizators}
Tā kā maksimālās atmiņas izmantošanas problēma ir tuva fragmentēšanai, tad autore nolēma realizēt divus analizatorus vienā gdb skriptā (sk. 5. pielikumu).
Skriptā ir nodefinēta komanda \texttt{analyze}, kura izsauc parējās komandas priekš fragmentēšanas un maksimālās atmiņas izmantošanas problēmas analizatoriem.
Šai komandai no gdb atkļūdotāja ir nepieciešams padot divus argumentus: galvenās arēnas adresi un skaitli, kurš norāda cik daļās būtu jāsadala kaudze.
Palaist skriptu var, izmantojot komandas, kuras parādītas attēlā 4.4.
Gdb skripts strādā uz 32 bitu datoru arhitektūras, taču to iespējams pielāgot arī 64 bitu arhitektūrai.
Skripts ir izstrādāts vienpavedienu lietotnei ar vienu galveno arēnu.

\begin{figure}[h]
\begin{lstlisting} [style=customgdb]
(gdb) source fragmentation.gdb
(gdb) p &main_arena
$1 = (struct malloc_state *) 0xb75ed440
(gdb) analyze 0xb75ed440 5
\end{lstlisting}
\caption{\textbf{\fontsize{11}{12}\selectfont {Gdb skripta palaišana}}}
\end{figure} %$


Šeit tiek aprakstīts maksimālās atmiņas izmantošanas problēmas analizators (sk. 5. pielikumu).
Atkarībā no otrā argumenta, kaudze tiks sadalīta vienā vai vairākos atsevišķos apgabalos.
Turpmāk darbā, viens kaudzes sadalījuma rezultāts tiks nosaukts par kaudzes apgabalu.
Katram kaudzes apgabalam tiks izrēķināta attiecība: kopējais atbrīvoto gabalu izmērs pret kopējo atbrīvoto un iedalīto gabalu izmēru.
Šī attiecība ļauj iegūt daļu, kuru aizņem atbrīvotie gabali katra kaudzes apgabalā.
Jā kaudzes apgabalam, kas atrodas blakus top, rādītājs ir lielāks par rādītāju pārējos kaudzes apgabalos, tad tās var liecināt par maksimālās atmiņas izmantošanas problēmu, kuras laikā atmiņa netiek atgriezta operētājsistēmai pēc maksimuma sasniegšanas.
Izdrukā (sk. 4.5. attēlu) blakus izrēķinātiem rādītājiem, tiek norādīta atmiņas gabala sākuma un beigu adreses.
Ja ir nepieciešams, tad var izdrūkat sīkāko kaudzes sadalījumu, norādot atbilstošo argumentu \texttt{analyze} komandai.
Piemēra redzamai programmai nav raksturīga maksimālās atmiņas izmantošanas problēma, jo atšķirība pēdējiem gabaliem nav lielāka par 5\%.

\begin{figure}[h]
\begin{lstlisting} [style=customgdb]
------------- Maksimālā atmiņas izmantošanas problēma --------
Atbrīvoto un iedalīto gabalu attiecība:
Apgabals 0x8279000 - 0x82b5238 35%
Apgabals 0x82b5238 - 0x82f32e0 31%
Apgabals 0x82f32e0 - 0x8333388 35%
Apgabals 0x8333388 - 0x8371430 31%
Apgabals 0x8371430 - 0x83a54b8 36%
\end{lstlisting}
\caption{\textbf{\fontsize{11}{12}\selectfont {Maksimālā atmiņas izmantošanas rādītājs}}}
\end{figure}

Algoritma realizācijai ir nepieciešams apstaigāt kaudzi un sakrāt datus par katru kaudzes apgabalu.
Sākumā tiek iegūta kaudzes sākuma adrese.
Lai to iegūtu, ir nepieciešams iegūt top gabala adresi, kaudzes izmēru un top gabala izmēru.
Top gabala adrese var tikt iegūta no top rādītāja, kurš atrodas galvenās arēnas struktūrā.
Kaudzes izmērs ir uzglabāts galvenās arēnas \texttt{system\_mem} elementā.
Top izmērs iegūstams piekļūstot top gabalam un nolasot \texttt{size} lauku.
Kaudze aug no mazākas adreses uz lielāko un top atrodas beigās.
Top gabala izmērs iekļauts kopējā kaudzes izmērā, bet šī atmiņa netiek izmantota programmā.
Tāpēc no top adreses vajag atņemt kaudzes izmēru un pieskaitīt top gabala izmēru.
Rezultātā tiek iegūta kaudzes sākuma adrese.
Sākot ar šo adresi tiek apstaigāti visi atmiņas gabali kaudzē.
Gabalu apstaigāšana notiek pieskaitot kaudzes adresei kārtēja gabala izmēru un saglabājot vajadzīgos datus.
Beigās tiek izdrukāts iegūtais rezultāts, kas izrēķināts karam kaudzes apgabalam.
Algoritms ir nodrošināts ar \texttt{div\_stat} komandas palīdzību.

Maksimālās atmiņas izmantošanas analizators sniedz statistiku, kura palīdz atrast problēmu.
Parauga programmai (sk. 2. pielikumu), kura tika palaista ar diviem argumentiem: 100, 100 skripts izvadīja, ka pēdējā kaudzes apgabalā atbrīvoto gabalu daļa no apgabala satura ir 98\%, ja kaudze tiek sadalītā 2 daļās.
Tā kā sniegtā izdruka atbilst programmas stāvoklim,tad tiek uzskatīts, ka dotās problēmas atkļūdošanai var izmantot atmiņas izmeti.

\section{Fragmentēšanas analizators}
Šajā sadaļā tiek aprakstīts fragmentēšanas analizators, kurš tika realizēts gdb skriptā (sk. 5. pielikumu).
Fragmentēšanas problēmas novērtēšanai analizators izvada divus svarīgākus rādītājus: lielāko un kopējo atbrīvoto gabalu izmēru arēnā.
Problēma būs novērojama, kad tiks iegūts tāds maksimālais atbrīvotais gabals, kurš nevar apmierināt pieprasījumu pēc atmiņas, ja kopējais izmērs ir pietiekošs.
Piemēra redzamai programmai (sk. 4.6. attēlu), ja programma pieprasīs vairāk nekā 8 kilobaitus un atmiņa netiks iedalīta no operētājsistēmas, tad tas varētu izraisīt sistēmas apstāšanos.
\begin{figure}[h]
\begin{lstlisting} [style=customgdb]
Bin numurs 1: 50 gabals(-li) (8196 - 8196 baiti), kopumā = 409800 baiti

------------------------ Fragmentēšana -----------------------
Lielākā gabala izmērs: 8196 baiti (8 KiB, 0 MiB),
Kopējā atbrīvotā atmiņa bin sarakstos: 409800 baiti (400 KiB, 0 MiB),
\end{lstlisting}
\caption{\textbf{\fontsize{11}{12}\selectfont {Fragmentēšanas rādītāji}}}
\end{figure}

Lai iegūtu abus šos rādītājus ir nepieciešams apstaigāt 128 bin sarakstus un no ikviena saraksta iegūt katra atmiņas gabala izmēru.
Parastie bin saraksti atrodas galvenajā arēnā, tāpēc piekļūt tiem var, ja ir zināma arēnas struktūra un tās sākuma adrese.
Gdb skriptā katram sarakstam ir numurs no 0 līdz 127 un apstaigāšana notiek, izmantojot nobīdes no galvenās arēnas sākuma un ņemot vērā kārtējā saraksta numuru.
Katrs gabals norāda uz nākamo un iepriekšējo atmiņas gabalu, tāpēc visus saraksta gabalus var apstaigāt, pārvietojoties pa sarakstu.
Pēdējais atmiņas gabals norāda uz kārtēja saraksta sākumu.
Saraksta apstaigāšana ir jābeidz, kad ir iegūta apstrādājamā saraksta sākuma adrese.
Katram gabalam tiek pārbaudīts vai tekošais gabala izmērs nav lielāks par maksimālo gabala izmēru sarakstā un tiek atjaunināta kopējā saraksta izmēru summa.
Apstrādājot iegūtās vērtības sarakstiem, tiek iegūti dotie rādītāji galvenajai arēnai.
Gabalu apstaigāšana sarakstā ir nodrošināta ar free\_chunk\_list komandu.

Pirms izvadīt lielāko un kopējo atbrīvoto gabalu izmēru rādītājus, tiek izdrukāta statistika par katru no bin sarakstiem (sk. 4.6. attēlu).
Tas palīdz iegūt detalizētu statistiku par visiem parastajiem sarakstiem, kuri nav tukši.
Statistika, kura tiek izdrukāta: skaits cik ir atbrīvoto gabalu sarakstā, amplitūda (mazākais gabals, lielākais gabals), kopējais gabalu izmērs sarakstā.

\begin{figure}[h]
\begin{lstlisting} [style=customgdb]
--------------------- Kopējā statistika ----------------------
Kaudzes segmenta izmērs: 1314816 baiti (1284 KiB, 1 MiB),
Kopējais programmai iedalītais atmiņas daudzums: 820208 baiti (800 KiB, 0 MiB),
Top gabala izmērs: 84808 baiti (82 KiB, 0 MiB),
Atbrīvoto gabalu skaits arēnā: 50,
\end{lstlisting}
\caption{\textbf{\fontsize{11}{12}\selectfont {Kopējā statistika}}}
\end{figure}

Skripts uzkrāj kopējo statistiku par kaudzi (sk. 4.7. attēlu). Izdrukas piemēra  ir redzams, ka kopumā tika iedalīti 50 atmiņas gabali.
Neizmantotās atmiņas daudzums (top gabala izmērs) ir 82 kilobaiti.
2/3 no iedalītās atmiņas kaudzē tiek iedalītas programmai ar malloc() vai lidzīgam funkcijam.






