\section{Atkļūdošanas algoritms}

 \begin{figure}[h]
\begin{center}
\includegraphics[scale=0.6]{method}
\end{center}
\caption{\textbf{\fontsize{11}{12}\selectfont {Uz metodi balstītā algoritma blok-shēma}}}
\end{figure}

Blok-shēmā (sk. 4.1. attēlu) ir attēlots kaudzes atkļūdošanas algoritms.
Šī algoritma ievaddati ir izpildāmā datne un atmiņas izmete.
Atmiņas izmete var būt bojāta, tāpēc pirms sākt atkļūdošanas procedūru ir nepieciešams veikt atmiņas izmetes validēšanu (1).
Šeit ir jāveic dažas pārbaudes, kuras varētu identificēt acīmredzamas kļūdas, piemēram, neatbilstību formātam vai atmiņas izmetes satura nepilnīgumu (pēc galvenes).
Ir iespējams, ka izpildāmā datne neatbilstis atmiņas izmetei (2), tas ir, kad tiek izmantota cita izpildāmā datne vai atmiņas izmetei neatbilst izpildāmas datnes versija.
Ir nepieciešams pārbaudīt atbilstību un jāpārliecinās, ka var piekļūt galvenās arēnas datiem.
Pēc dotām pārbaudēm var sākt atmiņas izmetes analīzi.
Blok-shēmā ir parādīts, ka ir nepieciešams darbināt analizatorus (3, 4, 5), kuri  izvada statistiku.
Daudzkodolu procesoriem ir iespējams realizēt algoritmu, kur analizatori strādās paralēli, vienkodolā visi analizatori izpildīsies secīgi. 
Analizatoru skaits nav ierobežots, taču bakalaura darba algoritms tiks nodemonstrēts ar 3 analizatoru piemēriem. 
Katrs analizators pārbauda noteiktās kaudzes problēmas pazīmes. 
Pēc tam tiek izvadīts kopējais rezultāts (6).
Ja ir zināma programmai raksturīga uzvedība, tad pēc rezultāta izvadīšanas var secināt pār problēmām.











\section{Secinājumi}
\begin{enumerate}
\item 
\end{enumerate}


