Šajā nodaļā ir aprakstīta metode, kura varētu tikt pielietota kaudzes atkļūdošanai. 
Šeit ir aplūkots metodes algoritms un trīs analizatoru realizācijas: atmiņas noplūdes analizators, maksimālās atmiņas izmantošanas problēmas analizators, fragmentēšanas problēmas analizators.
Realizētie analizatori ir uzskatāms demonstrējums, ka metode strādā un var tikt pielietota.
\section{Analizatora darbības princips}

 \begin{figure}[h]
\begin{center}
\includegraphics[scale=0.6]{method}
\end{center}
\caption{\textbf{\fontsize{11}{12}\selectfont {Uz metodi balstītā algoritma blokshēma}}}
\end{figure}

Blokshēmā (sk. 4.1. attēlu) ir attēlots kaudzes atkļūdošanas algoritms.
Šī algoritma ievaddati ir izpildāmā datne un atmiņas izmete.
Atmiņas izmete var būt bojāta, tāpēc, pirms sākt atkļūdošanas procedūru ir nepieciešams veikt atmiņas izmetes validēšanu (1).
Šeit ir jāveic vairākas pārbaudes, piemēram, pārbaudi uz neatbilstību formātam vai atmiņas izmetes nogriezto saturu.
Kā pārbaudīt vai atmiņas izmetei ir nogriezts saturs ir aprakstīts \ref{sec:validaty} sadaļā.
Nākamā pārbaude ir veikta pēc izpildāmās datnes nolasīšanas.
Ir iespējams, ka izpildāmā datne neatbilst atmiņas izmetei (2), tas var notikt divos gadījumos.
Pirmkārt, kad tiek izmantota cita izpildāmā datne atmiņas izmetes ģenerēšanai.
Otrkārt, kad atmiņas izmetei neatbilst izpildāmās datnes versija.
Ir nepieciešams pārbaudīt atbilstību un jāpārliecinās, ka var piekļūt galvenās arēnas datiem.
Visas iepriekš aprakstītās pārbaudes palīdz savlaicīgi uzzināt, ka atmiņas izmete nav derīga analīzei un pārtraukt algoritma darbību.
Pēc minētajām pārbaudēm var sākt atmiņas izmetes analīzi.
Blokshēmā ir parādīts, ka ir nepieciešams darbināt analizatorus, kuri  pārbauda problēmu pazīmes atmiņas izmetē.
Daudzkodolu procesoriem ir iespējams realizēt algoritmu, kur analizatori strādā paralēli, vienkodola procesoriem analizatori izpildīsies secīgi.
Analizatoru skaits nav ierobežots, taču bakalaura darbā algoritms tiks nodemonstrēts tiks nodemonstrēts realizējot un pārbaudot trīs analizatorus (3. atmiņas noplūdes analizators, 4. maksimālās atmiņas izmantošanas problēmas analizators, 5. fragmentēšanas problēmas analizators).
Katrs analizators pārbauda noteiktās kaudzes problēmas pazīmes.
Pēc tam tiek izvadīts kopējais rezultāts.
Ja ir zināma programmai raksturīgā uzvedība, tad pēc rezultātu izvadīšanas var secināt pār problēmām.

Pieeja tiek izstrādāta, ievērojot šādus ierobežojumus: %, kas aprakstīta bakalaura darbā,
\begin{itemize}
	\item GNU C bibliotēkas iedalītāja izmantošana;
    \item ELF atmiņas izmetes formāts;
    \item ir nepieciešams divreiz lielāks brīvas atmiņas daudzums nekā patērēts atkļūdojāmās programmas izpildei;
    \item analizatori neizdara galīgo slēdzienu par problēmu esamību.
\end{itemize}



\section{Atmiņas noplūdes analizators}
Autore realizēja atmiņas noplūdes analizatoru gdb skriptā (sk. 4. pielikumu). Skriptā ir definēta komanda \texttt{analyze}, kura izsauc parējās komandas.
Šai komandai no gdb atkļūdotāja ir nepieciešams padot vienu argumentu: galvenās arēnas adresi.
Atmiņas izmetei ir jābūt nosaukumam core un ir jāatrodas darba mapē.
Skripts darbosies uz 32 bitu datoru arhitektūras, kurā vismazāk nozīmīgie baiti tiek uzglabāti sākumā (little endian).
Skriptu ir iespējams pielāgot arī 64 bitu arhitektūrai un arhitektūrām ar visvairāk nozīmīgākiem baitiem sākumā (big endian).
Analizators ir izstrādāts vienpavedienu lietotnei ar vienu galveno arēnu.



Skripts atrod visus atmiņas gabalus kaudzē. 
Katram atmiņas gabalam tiek pārbaudīts stāvoklis: vai tas ir iedalīts programmai vai arī tas ir atbrīvots.
Šī informācija tiek iegūta, nolasot pēdējo P kontroles bitu nākamajam atmiņas gabalam.
Ja tekošais gabals ir atbrīvots, tad tas jau ir pievienots vienam no bin sarakstiem un to nevajag apstrādāt, tāpēc lai pārbaudītu nākamo atmiņas gabalu ir nepieciešams nobīdīt norādi.
Tādā veidā sākot ar kaudzes sākumu tiek apstrādāti visi atmiņas gabali, kuri ir novietoti pirms top gabala.
Katram iedalītajam gabalam kaudzē tiek izrēķināta lietotāja datu sākuma adrese.
Tā kā adrese ir iedalīta, tad procesa adrešu telpā ir jābūt norādei uz šo vietu atmiņā.
Ja norāde nav atrodama adrešu telpā, tad var uzskatīt, ka apgabals ir pazaudēts.
Šādi gadījumi var liecināt par atmiņas noplūdes problēmu.

\begin{figure}[h]
\begin{lstlisting} [style=customgdb]
---Type <return> to continue, or q <return> to quit---
Nav atrasta norāde uz adresi: 9c63498
Nav atrasta norāde uz adresi: 9c7c4a0
Nav atrasta norāde uz adresi: 9c954a8
Nav atrasta norāde uz adresi: 9cae4b0
Nav atrasta norāde uz adresi: 9cc74b8
Ir pazaudēti: 100 gabali.
\end{lstlisting}
\caption{\textbf{\fontsize{11}{12}\selectfont {Atmiņas noplūdes atrašana, gdb skripta izvads}}}
\end{figure} %$

Programmai iedalītās adreses tiek meklētas ar \texttt{grep} utilītprogrammas palīdzību.
Lai nodrošinātu adrešu meklēšanu visā procesa adrešu telpā, atmiņas izmetes saturs tiek saglabāts heksadecimalajā formātā atsevišķi izveidotajā datnē.
Šim nolūkam ir izmantota \texttt{od -t x} komanda, jo, atšķirībā no \texttt{xxd}, komanda ļauj bez baitu apgriešanas Intel x86 arhitektūrā iegūt korektas adreses, kurām  vismazāk nozīmīgākie baiti novietoti sākumā \cite{DPT}.
Atmiņas izmetes heksadecimalajā saturā tiek meklētas visas adreses, kuras tika iedalītas un uz kurām ir jābūt norādei.
Kad adrese ir atrasta, tad tiek meklēta nākamā iedalītā adrese, ja adrese netika atrasta, tad tiek atjaunināta pazaudēto gabalu statistika un izdrukāta pazaudētā atmiņas adrese.
Programmas izvads satur neatrasto norāžu skaitu un visas pazaudēto atmiņas gabalu adreses (sk. 4.2. attēlu).
Izmantojot pazaudēto atmiņas gabalu adreses var apskatīt datus, kuri uzglabāti pazaudētājos gabalos un uzzināt atmiņas gabala izmēru.
Pēc dotās informācijas var identificēt datus, kuri tika pazaudēti programmā.
%Realizējot šāda veida analizatoru, ir jāparedz gadījums, ka dati var netikt izlīdzināti 4 baitu robežai.
%Skriptā, šīm nolūkam visas adreses tika saplūdinātas.


\begin{figure}[h]
\begin{lstlisting} [language=C++]
char * str = (char *)malloc(sizeof(char) * num_elements) + 16; /* C */
\end{lstlisting}
\caption{\textbf{\fontsize{11}{12}\selectfont {Speciālgadījums - procesa adrešu telpā nav norādes uz gabala sākumu}}}
\end{figure}


Skripts var kļūdaini atrast pazaudētas adreses, ja gabaliem nodrošināta piekļuve ar nobīdi.
Tas notiek, kad tiek iedalīts gabals, bet no programmas nav norādes uz gabala sākumu.
Piemērā ir redzams, ka str norādei ir piešķirta atmiņas gabala adrese ar nobīdi (sk. 4.3. attēlu).
Šeit it nodemonstrēts viens no speciālajiem gadījumiem, kurš netiks apstrādāts skriptā. 
Turklāt ir iespējams iegūt adresi un apskatīties datus, kuri tiek uzglabāti atmiņā.



Atmiņas noplūdes analizators sameklē pazaudētos gabalus un ļauj secināt par problēmu no atmiņas izmetes satura.
Parauga programmai (sk. 1. pielikumu) tika padoti divi argumenti 100, 100. 
Analizators atgrieza rezultātu, ka tiek pazaudēti 100 atmiņas gabali.
Sniegtā izdruka atbilst programmas stāvoklim un pēc tās ir iespējams secināt, ka programma ir novērojama atmiņas noplūde.


\section{Maksimālās atmiņas izmantošanas problēmas analizators}
Tā kā maksimālās atmiņas izmantošanas problēma ir tuva fragmentēšanai, tad autore nolēma realizēt divus analizatorus vienā gdb skriptā (sk. 5. pielikumu).
Skriptā ir definēta komanda \texttt{analyze}, kura izsauc parējās komandas priekš fragmentēšanas un maksimālās atmiņas izmantošanas problēmas analizatoriem.
Šai komandai no gdb atkļūdotāja ir nepieciešams padot divus argumentus: galvenās arēnas adresi un skaitli, kurš norāda cik daļās būtu jāsadala kaudze.
Skriptu var palaist izmantojot komandas, kuras parādītas attēlā 4.4.
Gdb skripts strādā uz 32 bitu datoru arhitektūras, taču to iespējams pielāgot arī 64 bitu arhitektūrai.
Skripts ir izstrādāts vienpavedienu lietotnei ar vienu galveno arēnu.

\begin{figure}[h]
\begin{lstlisting} [style=customgdb]
(gdb) source fragmentation.gdb
(gdb) p &main_arena
$1 = (struct malloc_state *) 0xb75ed440
(gdb) analyze 0xb75ed440 5
\end{lstlisting}
\caption{\textbf{\fontsize{11}{12}\selectfont {Gdb skripta palaišana}}}
\end{figure} %$


Šeit tiek aprakstīts maksimālās atmiņas izmantošanas problēmas analizators (sk. 5. pielikumu).
Atkarībā no otrā argumenta, kaudze tiks sadalīta vienā vai vairākos atsevišķos apgabalos.
Turpmāk darbā viens kaudzes sadalījuma rezultāts tiks nosaukts par kaudzes apgabalu.
Katram kaudzes apgabalam tiks izrēķināta attiecība: kopējais atbrīvoto gabalu izmērs pret kopējo atbrīvoto un iedalīto gabalu izmēru.
Šī attiecība ļauj iegūt daļu, kuru aizņem atbrīvotie gabali katrā kaudzes apgabalā.
Ja kaudzes apgabalam, kas atrodas blakus top, rādītājs ir lielāks par rādītāju pārējos kaudzes apgabalos, tad tas var liecināt par maksimālās atmiņas izmantošanas problēmu, kuras laikā atmiņa netiek atgriezta operētājsistēmai pēc maksimuma sasniegšanas.
Izdrukā (sk. 4.5. attēlu) blakus izrēķinātajam rādītājam, tiek norādītas atmiņas gabala sākuma un beigu adreses.
Ja ir nepieciešams, tad var izdrukāt sīkāko kaudzes sadalījumu norādot atbilstošo argumentu \texttt{analyze} komandai.
Piemēra redzamai programmai nav raksturīga maksimālās atmiņas izmantošanas problēma, jo atšķirība pēdējiem gabaliem nav lielāka par 5\%.

\begin{figure}[h]
\begin{lstlisting} [style=customgdb]
------------- Maksimālā atmiņas izmantošanas problēma --------
Atbrīvoto gabalu attiecība pret atbrīvoto un iedalīto gabalu summu:
Apgabals 0x8279000 - 0x82b5238 35%
Apgabals 0x82b5238 - 0x82f32e0 31%
Apgabals 0x82f32e0 - 0x8333388 35%
Apgabals 0x8333388 - 0x8371430 31%
Apgabals 0x8371430 - 0x83a54b8 36%
\end{lstlisting}
\caption{\textbf{\fontsize{11}{12}\selectfont {Maksimālās atmiņas izmantošanas rādītājs}}}
\end{figure}

Algoritma realizācijai ir nepieciešams apstaigāt kaudzi un sakrāt datus par katru kaudzes apgabalu.
Gabalu apstaigāšana notiek, pieskaitot kaudzes adresei kārtējā gabala izmēru un saglabājot vajadzīgos datus: atmiņas gabala izmēru un stāvokli (atbrīvots vai iedalīts).
Ja atmiņas gabals pārsniedz kaudzes apgabala robežu, tad algoritmā tas tiks apstrādāts nākamajā kaudzes apgabalā.
Kad tiek iegūti visi atmiņas gabali, kuri pieder kaudzes apgabalam, tad tiek izdrukāta atbrīvoto gabalu daļa kārtējā kaudzes apgabalā.
Algoritms ir realizēts \texttt{div\_stat} komandā.

Maksimālās atmiņas izmantošanas analizators sniedz kaudzes apgabalu statistiku, kura palīdz atrast problēmu.
Parauga programmai (sk. 2. pielikumu), kura tika palaista ar diviem argumentiem: 100, 100, skripts atgrieza rezultātu, ka vienā kaudzes apgabalā atbrīvoto gabalu daļa no kopējā apgabala satura ir vienāda ar 99\%.
Kaudze tika sadalītā 2 daļās un otrajā kaudzes apgabalā šis rādītājs bija vienāds ar 0\%. 
Zinot to, ka visi atbrīvotie atmiņas gabali programmā tika saplūdināti vienā atmiņas gabalā un katrs gabals tiek ieskaitīts tieši vienā kaudzes apgabalā, tad var pareizi interpretēt iegūto rezultātu.
Pēc sniegtās izdrukas var izdarīt secinājumus par maksimālās atmiņas izmantošanas problēmu atmiņas izmetē.
Tātad tiek uzskatīts, ka dotās problēmas identificēšanai var izmantot atmiņas izmeti.

\section{Fragmentēšanas analizators}
Šajā sadaļā tiek aprakstīts fragmentēšanas analizators, kurš tika realizēts gdb skriptā (sk. 5. pielikumu).
Fragmentēšanas problēmas novērtēšanai analizators izvada divus svarīgākus rādītājus: lielāko un kopējo atbrīvoto gabalu izmēru arēnā.
Problēma būs novērojama, kad tiks iegūts tāds maksimālais atbrīvotais gabals, kurš nevar apmierināt pieprasījumu pēc atmiņas, ja kopējais izmērs ir pietiekošs.
Piemēra redzamai programmai (sk. 4.6. attēlu), ja programma pieprasīs vairāk nekā 8 kilobaitus un top gabala izmērs ir mazāks par doto vērtību, tad atmiņa var netikt iedalīta no operētājsistēmas un tas varētu izraisīt sistēmas apstāšanos.
\begin{figure}[h]
\begin{lstlisting} [style=customgdb]
Bin numurs 1: 50 gabals(-li) (8196 - 8196 baiti), kopumā = 409800 baiti

------------------------ Fragmentēšana -----------------------
Lielākā gabala izmērs: 8196 baiti (8 KiB, 0 MiB),
Kopējā atbrīvotā atmiņa bin sarakstos: 409800 baiti (400 KiB, 0 MiB),
\end{lstlisting}
\caption{\textbf{\fontsize{11}{12}\selectfont {Fragmentēšanas rādītāji}}}
\end{figure}

Lai iegūtu abus šos rādītājus, ir nepieciešams apstaigāt 128 bin sarakstus un no ikviena saraksta iegūt katra atmiņas gabala izmēru.
Parastie bin saraksti atrodas galvenajā arēnā, tāpēc piekļūt tiem var, ja ir zināma arēnas struktūra un tās sākuma adrese.
Gdb skriptā katram sarakstam ir numurs no 0 līdz 127 un apstaigāšana notiek, izmantojot nobīdes no galvenās arēnas sākuma un ņemot vērā kārtējā saraksta numuru.
Katrs gabals norāda uz nākamo un iepriekšējo atmiņas gabalu, tāpēc visus saraksta gabalus var apstaigāt, pārvietojoties pa sarakstu.
Pēdējais atmiņas gabals norāda uz kārtējā saraksta sākumu.
Saraksta apstaigāšana ir jābeidz, kad ir iegūta apstrādājamā saraksta sākuma adrese.
Katram gabalam tiek pārbaudīts vai tekošais gabala izmērs nav lielāks par maksimālo gabala izmēru sarakstā, un tiek atjaunināta kopējā saraksta izmēru summa.
Apstrādājot iegūtās vērtības sarakstiem tiek iegūti dotie rādītāji galvenajai arēnai.
Algoritms ir realizēts \texttt{free\_chunk\_list} komandā.

Pirms izvadīt lielāko un kopējo atbrīvoto gabalu izmēru rādītājus, tiek izdrukāta statistika par katru no bin sarakstiem (sk. 4.6. attēlu).
Tas palīdz iegūt informāciju par katru parasto sarakstu, kurš nav tukšs.
Sarakstiem tiek izdukāta šāda informācija: atbrīvoto gabalu daudzums sarakstā, amplitūda (mazākais gabals, lielākais gabals), kopējais gabalu izmērs sarakstā.



\begin{figure}[h]
\begin{lstlisting} [style=customgdb]
--------------------- Kopējā statistika ----------------------
Kaudzes segmenta izmērs: 1314816 baiti (1284 KiB, 1 MiB),
Kopējais programmai iedalītais atmiņas daudzums: 820208 baiti (800 KiB, 0 MiB),
Top gabala izmērs: 84808 baiti (82 KiB, 0 MiB),
Atbrīvoto gabalu skaits arēnā: 50,
\end{lstlisting}
\caption{\textbf{\fontsize{11}{12}\selectfont {Kopējā statistika}}}
\end{figure}

Skripts uzkrāj arī kopējo statistiku par kaudzi (sk. 4.7. attēlu). Izdrukas piemērā  ir redzams, ka kopumā tika atbrīvoti 50 atmiņas gabali.
Neizmantotās atmiņas daudzums (top gabala izmērs) ir 82 kilobaiti.
2/3 no iedalītās atmiņas kaudzē tiek iedalītas programmai ar malloc() vai līdzīgām funkcijām.
Kopējā statistika tiešā veida neliecina par fragmentēšanas problēmu, bet palīdz novērtēt kopējo atmiņas stāvokli.
Piemēram, liels atbrīvoto gabalu skaits nav fragmentēšanas problēmas cēlonis gadījumos, kad katrs pieprasījums pēc atmiņas var tikt apmierināts.
Bet, ja divi galvenie fragmentēšanas rādītāji liecina, ka ir fragmentēšanas problēma, tad atbrīvoto gabalu skaits var apstiprināt šo pieņēmumu.
Top gabala izmērs ir neizmantotās atmiņas izmērs, kurš nodrošina pēdējo iespēju iedalīt atmiņu.
Ja ir redzams, ka top gabalam ir mazs izmērs\footnote{Tāds, kurš nevar apmierināt pieprasījumus pēc atmiņas.}, tad tas nozīme, ka dinamiskā atmiņas vairs netiek iedalīta programmai un tas varēja izraisīt programmas apstāšanos.

Parauga fragmentēšanas programmai (sk. 3. pielikumu) tika padoti divi argumenti 100, 100. 
Skripts atgrieza rezultātu, ka lielāka gabala izmērs ir 104 baiti, bet kopējā atbrīvotā atmiņa bin sarakstos ir vienāda ar 5200 baitiem.
Sniegtā izdruka atbilst programmas stāvoklim un, zinot atmiņas daudzumu, ko pieprasa programma, izdruka ļaus secināt par doto problēmu.
Tāpēc tiek uzskatīts, ka dotās problēmas identificēšanai var izmantot atmiņas izmeti.





