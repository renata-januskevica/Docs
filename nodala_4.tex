\section{Analizatora darbības princips}

 \begin{figure}[h]
\begin{center}
\includegraphics[scale=0.6]{method}
\end{center}
\caption{\textbf{\fontsize{11}{12}\selectfont {Uz metodi balstītā algoritma blok-shēma}}}
\end{figure}

Blok-shēmā (sk. 4.1. attēlu) ir attēlots kaudzes atkļūdošanas algoritms.
Šī algoritma ievaddati ir izpildāmā datne un atmiņas izmete.
Atmiņas izmete var būt bojāta, tāpēc pirms sākt atkļūdošanas procedūru ir nepieciešams veikt atmiņas izmetes validēšanu (1).
Šeit ir jāveic dažas pārbaudes, kuras varētu identificēt kļūdas, piemēram, neatbilstību formātam vai atmiņas izmetes nogriezto saturu, šīs atmiņas izmetes stāvoklis ir sīkāk aprakstīts \ref{sec:validaty} sadaļā.
Ir iespējams, ka izpildāmā datne neatbilstis atmiņas izmetei (2), tas ir, kad tiek izmantota cita izpildāmā datne atmiņas izmetes ģenerēšanai vai atmiņas izmetei neatbilst izpildāmas datnes versija.
Ir nepieciešams pārbaudīt atbilstību un jāpārliecinās, ka var piekļūt galvenās arēnas datiem.
Pēc dotām pārbaudēm var sākt atmiņas izmetes analīzi.
Blok-shēmā ir parādīts, ka ir nepieciešams darbināt analizatorus, kuri  pārbauda problēmu pazīmes atmiņas izmetē.
Daudzkodolu procesoriem ir iespējams realizēt algoritmu, kur analizatori strādās paralēli, vienkodolā visi analizatori izpildīsies secīgi. 
Analizatoru skaits nav ierobežots, taču bakalaura darba algoritms tiks nodemonstrēts ar 3 analizatoru piemēriem (3. atmiņas noplūdes analizators, 4. maksimālās atmiņas izmantošanas problēmas analizators, 5. fragmentēšanas problēmas analizators). 
Katrs analizators pārbauda noteiktās kaudzes problēmas pazīmes. 
Pēc tam tiek izvadīts kopējais rezultāts (6).
Ja ir zināma programmai raksturīga uzvedība, tad pēc rezultāta izvadīšanas var secināt pār problēmām.

Pieeja tiek izstrādāta ievērojot sekojošus ierobežojumus: %, kas aprakstīta bakalaura darbā,
\begin{itemize}
	\item GNU C bibliotēkas izmantošana sākot ar 2.3 versiju;
    \item a.out vai ELF atmiņas izmetes formāts;
    \item analizātori nedod pilnīgu secinājumu par problēmas esamību, bet sniedz informāciju par sistēmas stāvokli, kura ļauj izstradatājam secināt par problēmu.
\end{itemize} 



\section{Atmiņas noplūdes analizators}

\section{Maksimālās atmiņas izmantošanas problēmas analizators}
Tā kā maksimālās atmiņas izmantošanas problēma ir tuva fragmentēšanai, tad autore nolēma realizēt divus analizatorus vienā gdb skriptā.
Šajā sadaļā tiek aprakstīts maksimālās atmiņas izmantošanas problēmas analizators (sk. 4. pielikumu).
Skriptā, kurš realizē divu problēmu analizatorus, ir nodefinēta komanda \texttt{analyze}, kura izsauc parējās komandas.
Šai komandai no gdb atkļūdotāja ir nepieciešams padot 2 argumentus: galvenās arēnas adresi un skaitli, kurš norāda cik sīki tiks sadalīta kaudze.
Par problēmu liecinās atbrīvoto un iedalīto gabalu attiecība, kura tiek izrēķināta katram apgabalam kaudzē.
Kaudzes sadalījuma lielumu noteic lietotājs, tāpēc ir iespējams izdrukāt 



\section{Fragmentēšanas analizators}
Šajā sadaļā tiek aprakstīts fragmentēšanas analizators, kurš tika realizēts gdb skriptā (sk. 4. pielikumu).
Fragmentēšanas problēmas novērtēšanai analizators izvada 2 svarīgākus rādītājus (sk. 4.2. attēlu): lielāko un kopējo atbrīvoto gabalu izmēru arēnā.
Problēma būs novērojama, kad tiks iegūts tāds maksimālais atbrīvotais gabals, kurš nevar apmierināt pieprasījumu pēc atmiņas, ja kopējais izmērs ir pietiekošs.

\begin{figure}[h]
\begin{lstlisting} [style=customgdb]
(gdb) analyze 0xb75ed440 7
Bin numurs 1: 50 gabals(-li) (8196 - 8196 baiti), kopumā = 409800 baiti

------------------------ Fragmentēšana -----------------------
Lielākā gabala izmērs: 8196 baiti (8 KiB, 0 MiB),
Kopējā atbrīvotā atmiņa bin sarakstos: 409800 baiti (400 KiB, 0 MiB),
\end{lstlisting}
\caption{\textbf{\fontsize{11}{12}\selectfont {Fragmentēšanas rādītāji}}}
\end{figure}

Lai iegūtu abus šos rādītājus ir nepieciešams apstaigāt 128 bin sarakstus un no ikviena saraksta iegūt katra atmiņas gabala izmēru.
Parastie bin saraksti atrodas galvenajā arēnā, tāpēc piekļūt tiem var, ja ir zināma arēnas struktūra un tās sākuma adrese.
Gdb skriptā katram sarakstam ir numurs no 0 līdz 127 un apstaigāšana notiek, izmantojot nobīdes no galvenās arēnas sākuma un ņemot vērā kārtējā saraksta numuru.
Katrs gabals norāda uz nākamo un iepriekšējo atmiņas gabalu, tāpēc visus saraksta gabalus var apstaigāt, pārvietojoties pa sarakstu.
Pēdējais atmiņas gabals norāda uz kārtēja saraksta sākumu.
Saraksta apstaigāšana ir jābeidz, kad ir iegūta apstrādājamā saraksta sākuma adrese.
Katram gabalam tiek pārbaudīts vai tekošais gabala izmērs nav lielāks par maksimālo gabala izmēru sarakstā un tiek atjaunināta kopējā saraksta izmēru summa.
Gabalu apstaigāšana sarakstā ir nodrošināta ar free\_chunk\_list komandu.

Pirms izvadīt lielāko un kopējo atbrīvoto gabalu izmēru rādītājus, tiek izdrukāta statistika par katru no bin sarakstiem. 
Tas palīdz iegūt detalizētu statistiku par visiem parastajiem sarakstiem, kuri nav tukši.
Tas ir: skaits cik ir atbrīvoto gabalu sarakstā, amplitūda (mazākais gabals, lielākais gabals), kopējais gabalu izmērs sarakstā.


\section{Secinājumi}
\begin{enumerate}
\item 
\end{enumerate}


