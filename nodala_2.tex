\section{Atmiņas iedalīšanas paņēmieni}
%\label{sect:Motivation}

Pirms izpildīt programmu, operētājsistēmai ir nepieciešams iedalīt resursus, tādus kā atmiņas adreses.  
Eksistē divas atmiņas iedalīšanas paņēmieni: statiskā un dinamiskā atmiņas iedalīšana (sk. 2.1. attēls). 

\begin{figure}[h]
\begin{center}
\includegraphics[scale=0.25]{alloc}
\end{center}
\caption{\textbf{\fontsize{11}{12}\selectfont {Atmiņas iedalīšanas paņēmienu klasifikācija}}}
\label{fig:alloc}
\end{figure}


\subsubsection{Statiskā atmiņas iedalīšana}
Statiskā atmiņas iedalīšana nozīme, ka atmiņa tiek iedalīta vienu reizi pirms programmas palaišanas, parasti tas notiek kompilācijas laikā.
Programmas izpildēs laikā atmiņa vairs netiek iedalīta, ka arī netiek atbrīvota. 
Statiskais atmiņas iedalīšanas paņēmiens nodrošina to, ka atmiņa tiek iedalīta statiskiem un globāliem mainīgiem, neatkarībā no tā vai mainīgais tiks izmantots programmā vai nē \cite{mem_alloc}.

\subsubsection{Dinamiskā atmiņas iedalīšana}
Dinamiskā atmiņas iedalīšana nozīme, ka atmiņa tiek iedalīta programmas izpildes laikā.
Tas var būt nepieciešams, kad atmiņas daudzums nav zināms programmas kompilācijas laikā. 
Dinamiskā atmiņas iedalīšana, var būt realizēta ar steka vai kaudzes palīdzību un var būt automātiskā vai kontrolētā \cite{SDMA}.

Automātiskā iedalīšana notiek, kad sākās programmas funkcijas izpilde. 
Šeit viens un tas pats atmiņas apgabals, kurš bija atbrīvots, var tikt izmantots  vairākas reizēs. 
Piemēram, kad tekošās funkcijas argumenti un lokālie mainīgie ir saglabāti stekā un izdzēsti pēc šīs funkcijas izpildes.
Pēc tam atbrīvotā atmiņa var būt izmantota atkārtoti. 
Priekš automātiskās atmiņas iedalīšanas izmato steku.
Visiem funkcijas mainīgiem var piekļūt izmantojot steka norādes nobīdi, kas tiek uzglabāta reģistrā, piemēram,  
Intel x86 procesoros, 16 bitu režīmā reģistrs ir \texttt{SP}, 32 bitu režīmā - \texttt{ESP} un 64 bitu režīmā - \texttt{RSP} \cite{JCL}.
Reģistrs uzglabā adresi, kurā atrodas pēdējā uzglabāta vērtība stekā.
Steka pārpildīšana var notikt dažādu iemeslu dēļ, piemēram to var izraisīt dziļa rekursija.
 
 Kontrolētā atmiņas iedalīšana nozīme, ka programma var izvēlēties patvaļīgus, brīvus atmiņas apgabalus priekš programmas datiem. 
Kontrolētā jeb manuālā atmiņas iedalīšana ir realizēta ar kaudzes palīdzību. 
Šeit nav iespējams piekļūt datiem izmantojot vienu norādi un tās nobīdi. 
Tagad katram izdalītam atmiņas apgabalam var piekļūt tikai tad, ja ir norāde uz šo izdalīto atmiņas apgabalu. 
Gadījumos, kad norādes nav, tad adreses vairāk nav sasniedzamas un kļūst pazaudētas.


\section{Atmiņas pārvaldība}
Kad tiek izpildīta jebkura programma, atmiņa tiek pārvaldīta divos veidos: ar kodola palīdzību vai ar lietotnes funkciju izsaukumiem, tādiem kā malloc().

\subsection{Kodola atmiņa}
Operētājsistēmas kodols pārvalda  visus atmiņas pieprasījumus, kas attiecās uz programmu vai programmas instancēm.
Kad lietotājs sāk programmas izpildi, tad kodols iedala atmiņas apgabalu tekošai programmai.
Pēc tam programma pārvalda iedalīto apgabalu, sadalot to vairākos segmentos: 
%\vspace{-20 pt}
\begin{itemize}
	\item   Teksts - uzglabāti dati, kuri tiek izmantoti tikai lasīšanai. Tās ir koda instrukcijas. 
    Vairākas programmas instances var izmantot šo atmiņas apgabalu.
    \item Statiskie dati - apgabals, kurā tiek uzglabāti dati ar iepriekš zināmu izmēru. Tās ir globālie un statiskie mainīgie.
    Operētājsistēma iedala šī apgabala kopiju priekš katras programmas instances atsevišķi.
    \item Atmiņas arēna - apgabals, kurā tiek uzglabāta dinamiski iedalītā atmiņa. Atmiņas arēna sastāv no kaudzes un neizmantotās atmiņas.
    Kaudze ir apgabals, kurā atrodas visa lietotāja iedalīta atmiņa programmas izpildei.
    \item Steks - apgabals, kurā tiek uzglabāts funkciju izsaukumu stāvoklis, katram funkcijas izsaukumam. Steks aug no lielākas adreses uz mazāko. 
    Unikāla atmiņas arēna un steks iedalīti priekš katras programmas instances atsevišķi.
\end{itemize} 
 Lai palielinātu atmiņas arēnas izmēru, tiek veikts brk() sistēmas izsaukums. Izsaukums pieprasa papildus atmiņu no kodola.
 Iedalīto adrešu intervālu stekam un atmiņas arēnai var atrast \texttt{/proc/<pid>/maps} datnē.
\subsection{Lietotāja atmiņa}
Lietotāju iedalīta atmiņa atrodas kaudzē, kura tiek novietota atmiņas arēnā. 
Atmiņās arēna C valodā tiek pārvaldīta ar malloc(), realloc(), free() un calloc() funkciju palīdzību \cite {atparv}.
C++ valodā ir izmantots operators new, lai pieprasītu atmiņu.   
Attēlā 2.2. ir redzama C un C++ sintakse atmiņas pieprasīšanai izmantojot C un C++ kodu.
Vienīgais arguments malloc() funkcijai ir baitu skaits.
C programmai, lai saskaitītu cik baitu ir nepieciešams pieprasīt, ir nepieciešams zināt cik daudz vietas aizņem viens elements un kāds ir elementu skaits.
Funkcija malloc() atgriež void tipa rādītāju, tāpēc C programmās ir nepieciešams izmantot drošo tipa pārveidotāju (typecast). 
Tas ir nepieciešams, lai saglabātu atgriezto norādi lokālajā mainīgajā. Atmiņas inicializācija C kodā
var būt veikta izmantojot arī citas funkcijas, piemēram calloc() funkciju, kura atgriež atmiņas apgabalu inicializētu ar 0 vērtībām.

\begin{figure}[h]
\begin{lstlisting}
int * ptrl = new int; // C++
int * ptrl = (int *)malloc( sizeof(int) ); /* C */

char * str = new char[num_elements]; // C++
char * str = (char *)malloc( sizeof(char) * num_elements ); /* C */
\end{lstlisting}
\caption{\textbf{\fontsize{11}{12}\selectfont {Dinamiskās atmiņas iedalīšana C un C++}}}
\end{figure}


Funkcija free() atbrīvo ar malloc() palīdzību iedalīto atmiņu.
Lielāka atšķirība starp free() un delete ir tāda, ka vecajās free() sistēmās netiek nodrošināts atbalsts free() funkcijai, kad arguments ir null \cite{POCF}. 

Programmas rakstīšanā nejauc kopā C un C++ stilus, tāpēc priekš C++ programmas izmanto 
new un delete operatorus (sk. 2.3. attēls), bet priekš C programmām malloc() un free.


\begin{figure}[h]
\begin{lstlisting}
delete ptrl; // C++

If( ptrl != NULL )
	free(ptrl); /* C */
\end{lstlisting}
\caption{\textbf{\fontsize{11}{12}\selectfont {Dinamiskās atmiņas atbrīvošana C un C++}}}
\end{figure}


Ja atmiņa pēc izmantošanas netiek nekad atbrīvota, un katru reizi, izpildot vienu un to pašu koda gabalu, iedalīta no jauna, tad pieejams operētājsistēmai atmiņas daudzums ar laiku samazinās.
Sākumā sistēma paliek arvien lēnāka, pēc tam parasti notiek sistēmas apstāšanās.






\section{Atmiņas iedalīšana glibc bibliotēkā}
Darbā tiks aplūkota GNU C bibliotēkas ptmalloc2 realizācija, kuru izstrādāja Wolfram Gloger, balstoties uz Doug Lea dlmalloc realizāciju. 
Atšķirībā no dlmalloc, ptmalloc2 izmanto atsevišķas arēnas priekš pavedieniem.
Tāpēc atmiņas iedalīšana var notikt vienlaicīgi vairākos pavedienos.
GNU C bibliotēka iekļauj ptmalloc2 realizāciju sākot ar 2.3 versiju \cite {MWIKI}. 

\subsection{Atmiņas chunk gabali}
Atmiņa no kaudzēs tiek iedalīta, izmantojot chunk struktūru.
Katru reizi ir iedalīts lielāks atmiņās gabals nekā pieprasīts ar malloc() funkciju.
Tas ir nepieciešams, lai varētu saglabāt uzturēšanai nepieciešamo informāciju. Iedalītam gabalam uzturēšanas informācija ir vienāda ar 8 vai 16 baitiem.
Uzturēšanas informācijai ir izmantoti divas \texttt{size\_t} tipa vērtības un, ja gabals ir atbrīvots, tad divi rādītāji uz \texttt{malloc\_chunk} struktūru. 
Otrs iemesls kāpēc ir iedalīts lielāks atmiņas daudzums ir izlīdzināšana skaitlim, kas ir 2*sizeof(INTERNAL\_SIZE\_T) reizinājums (8 baitu izlīdzinājums ar 4 baitu INTERNAL\_SIZE\_T) \cite {MALLOC}.  


\begin{figure}[h]
\begin{center}
\includegraphics[scale=0.21]{chunks}
\end{center}
\caption{\textbf{\fontsize{11}{12}\selectfont {Atmiņas gabalu struktūra}}}
\label{fig:chunks}
\end{figure}

Atmiņas gabala fiziska struktūra ir vienāda gan normāliem (normal chunk), gan ātriem gabaliem (fast - chunk), bet ir atkarīga no stāvokļa un var tikt interpretēta dažādi (sk. 2.4. attēls).
No kreisās pusēs attēlots atmiņas gabals, kurš bija iedalīts procesam, no labās, tās, kurš bija atbrīvots.
Abos gadījumos rādītājs chunk attēlo atmiņas gabalu sākumu. Pēc šī radītāja var iegūt iepriekšēja gabala izmēru, ja iepriekšējais gabals bija atbrīvots.
Gadījumā, kad iepriekšējais gabals ir iedalīts, tad pēc chunk rādītāja atrodas daļa no iepriekšēja gabala lietotāja datiem. 
Pēc tam seko kārtēja gabala izmērs un 3 biti ar meta informāciju. 
Tā kā notiek izlīdzināšana 2*sizeof(INTERNAL\_SIZE\_T)  reizinājumam, tad 3 pēdējie biti netiek izmantoti izmēra glabāšanai. 
Šos bitus izmanto kontroles zīmēm.
Biti palīdz noteikt vai kārtējais gabals nepieder main arēnai, vai apgabals tiek iedalīts ar mmap() sistēmas izsaukumu un vai iepriekšējais atmiņas gabals tiek izmantots (A, M un P atbilstoši).
Rādītājs mem ir malloc() funkcijas atgriežamā vērtība. Iedalīts apgabals stiepjas līdz atmiņas gabala struktūras beigām.
Pēc šī rādītāja var tikt uzglabāti dati, kad atmiņa ir iedalīta un, ja tā ir atbrīvota, tad šeit tiks uzglabāti divi radītāji uz nākamo un iepriekšējo atbrīvotiem gabaliem, kas atrodas sarakstā. 
 
 \subsection{Bin saraksti ptmalloc2 versijā}
Ja atmiņa bija atbrīvota, tad tā tiks uzglabāta saistītajā sarakstā uz kuru norāda bin masīvs.
Sarakstā gabali ir sakārtoti, lai varētu ātrāk atrast piemērotu atmiņas gabalu.
Atbrīvots atmiņas apgabals netiek atgriezts operētājsistēmai, bet ir defragmentēts vai sapludināts ar pārējiem gabaliem un ievietots sarakstā, lai tiktu vēlreiz iedalīts. Eksistē divi bin saraksti: fastbin un normal bin.

Atmiņas gabali, kuri paredzēti fastbin glabāšanai ir mazi. Pēc noklusējuma atmiņas gabalu izmērs ir 64 baiti, bet to var palielināt līdz 80 baitiem \cite {MALLOC}. 
Atmiņas gabali atrodas vienvirzienu sarakstā un nav sakārtoti. 
Lai samazinātu fragmentācijas iespējamību, programma, kad pieprasa vai atbrīvo lielus atmiņas gabalus mēdz sapludināt atmiņas gabalus, kuri atrodas fastbin sarakstā.
Piekļuve tādiem atmiņas gabaliem ir ātrāka nekā piekļuve normāliem gabaliem. 
Fastbin saraksta elementi ir apstrādāti pēdējais iekšā pirmais āra (jeb LIFO) kārtībā \cite {Binning}.
Kad tiek pieprasīta atmiņas daudzums, kas mazāks par 64 baitiem, tad tas tiek atgriezts konstantā laikā  \cite {ACCA}.

Normāla izmēra gabali var būt sadalīti 3 veidos. Pirmkārt, bin saraksts uzglabā nesakārtotus gabalus, kuri nesen bija atbrīvoti.  
Pēc tam tie tiks novietoti vienā no atlikušiem bin sarakstiem: mazā vai lielā izmēra. 
Mazā izmēra saraksts uzglabā atmiņas gabalus, kuri ir mazāki par 512 baitiem. 
Vairāki ātrie gabali var būt sapludināti un uzglabāti dotajā sarakstā. 
Lielā izmēra saraksts uzglabā atmiņas gabalus, kuri ir lielāki par 512 baitiem, bet mazāki par 128 kilobaitiem. 
Gabali, kuru izmērs ir lielāks par 128 kilobaitiem tiek iedalīti, izmantojot mmap().
Lielā izmēra sarakstā elementi ir sakārtoti pēc izmēra un ir iedalīti pirmais iekšā, pirmais ārā (jeb FIFO) kārtībā \cite {Binning}. 
Eksistē divi citi atmiņas gabali (top chunk un last\_remainder), kuriem ir īpaša nozīme un tie netiek uzglabāti bin sarakstos. 

Top chunk ir atmiņas gabals, kurš ierobežo pieejamās atmiņas daudzumu.
Tas ir izmantots gadījumos, kad nav piemērotu gabalu bin sarakstos, kuri apmierina pieprasījumu vai varētu būt saplūdināti, lai apmierinātu pieprasījumu.
Top chunk nodrošina pēdējo iespēju iedalīt pieprasīto atmiņas daudzumu.
Top chunk var mainīt savu izmēru. Tas saraujas, kad atmiņa ir iedalīta un izstiepjas, kad atmiņa ir atbrīvota blakus top chunk struktūrai. 
Ja ir pieprasīta atmiņa, kas ir lielāka par pieejamo, tad top chunk var paplašināties ar brk() palīdzību.
Top chunk ir līdzīgs jebkuram citam atmiņas apgabalam. 
Galvenā atšķirība ir lietotāja datu sekcija, kura netiek izmantota, ka arī speciāla top chunk apstrāde, lai nodrošinātu, ka top chunk vienmēr eksistē.

Last\_remainder ir vel viens atmiņas gabals ar īpašu nozīmi.
Tas ir izmantots gadījumos, kad ir pieprasīts mazs atmiņas gabals, kas neatbilst nevienam bin saraksta elementam. 
Last\_remainder ir dalījuma atlikums, kurš izveidojās pēc lielāka gabala sadalīšanas, lai apmierinātu pieprasījumu pēc maza gabala  \cite {BLACKHAT}.

\subsection{Atmiņas arēna}

Lai uzlabotu veiktspēju vairākpavedienu procesiem, GNU C bibliotēkā tiek izmantotas vairākas atmiņas arēnas. 
Katrs funkcijas malloc() izsaukums bloķē izmantoto arēnu. Laikā, kad arēna ir nobloķēta notiek atmiņas apgabala iedalīšana.
Kad vairākiem pavedieniem ir nepieciešams vienlaicīgi iedalīt atmiņu no kaudzes, arēnas bloķēšana var būtiski samazināt veiktspēju.
Gadījumos, kad pavedieni izmanto atmiņu no vairākām atsevišķām arēnām, tad vienas arēnas bloķēšana neietekmē atmiņas iedalīšanu parējās arēnās un atmiņas iedalīšana var notikt paralēli.
GNU C bibliotēkā darbība ar arēnām notiek saskaņā ar sekojošo algoritmu: 

\begin{enumerate}
\item malloc() izsaukums vēršas pie arēnas, kurai piekļuva iepriekšējo reizi,
\item ja arēna ir nobloķēta, tad malloc() vēršas pie nākamas izveidotās arēnas,
\item ja nav piekļuves nevienai arēnai, tad tiek izveidota jauna arēna un malloc() vēršas pie tās.
\end{enumerate}
Vispirms atmiņas iedalīšana sākas no galvenās arēnas (main arena). 
\begin{figure}[h]
\begin{center}
\includegraphics[scale=0.50]{threads}
\end{center}
\caption{\textbf{\fontsize{11}{12}\selectfont {Arēnas GNU C bibliotēkā }}}
\end{figure}

GNU C bibliotēkā ir globāls \texttt{malloc\_state} objekts - globāla arēna, kura atšķiras ar to, kā atmiņu no kodola iegūst, izmantojot brk() (sk. 2.5. attēls) \cite{AMM}. 
Pārējas arēnas šīm nolūkam izmanto \texttt{mmap()} izsaukumu. 
Ja kārtēja kaudze ir izlietota, tad tiek iedalīta jauna kaudze ar fiksēto 64 MB izmēru.
Tāda veidā arēnas var tikt paplašinātas, pievienojot jaunās kaudzes un savienojot tās sava starpā.
Lai nodrošinātu labāku veiktspēju, tiek izmantots modelis: katram pavedienam - viena arēna. 
Ja malloc() pirmo reizi izsaukts pavedienā, tad neatkarībā no tā vai arēna bija nobloķēta vai nē, tiks izveidota jauna arēna.
Arēnu skaits ir ierobežots atkarībā no kodolu skaita, 32 bitu vai 64 bitu arhitektūras un mainīga MALLOC\_ARENA\_MAX vērtības.
Tā kā pavedienu skaits parasti nepārsniedz divkāršo kodolu skaitu, tad normālā gadījumā katrs pavediens izmanto atsevišķo arēnu. 



