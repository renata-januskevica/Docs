\noindent 

Debugging - Atkļūdošana - Procedūra pieļauto kļūdu atrašanai, lokalizēšanai un novēršanai.

POSIX (Portable Operating System Interface) - IEEE un ISO standartu kopa,  kas reglamentē kā rakstīt pieteikumu pirmkodu tā, lai lietotne būtu pārnēsājama starp operētājsistēmām.

IEEE (Institute of Electrical and Electronics Engineers) - Elektrotehnikas un elektronikas inženieru institūts.

ISO (International Organization for Standardization) - Starptautiskā Standartu organizācija.

Hard link - Stingrā saite - rādītājs uz datnes indeksa deskriptoru. 

Segment - segments - blakusiedalītas atmiņas reģions.

Heap - Kaudze - globāla datu struktūra,  no kuras tiek iedalīta dinamiskā atmiņa procesam.

ELF (Executable and Linkable Format) - ELF formāts -  bināro datņu formāts, kurš ir Unix un Linux standarts. Šīs formāts var būt izmantots priekš izpildāmām datnēm, objektu datnēm, bibliotēkām un atmiņas izmetēm.

Memory allocation - Atmiņas iedalīšana - atmiņās adreses piesaistīšana instrukcijām un datiem.

Static memory allocation - Statiskā atmiņas iedalīšana -  atmiņas iedalīšanas paņēmiens, kurš ir pielietots kompilācijas laikā.

Dynamic memory allocation - Dinamiskā atmiņas iedalīšana - atmiņas iedalīšanas paņēmiens, kurš pielietots programmas izpildes laikā. 

Core dump - Atmiņas izmete - visa atmiņas satura vai tā daļas pārrakstīšana citā vidē (parasti - no iekšējās atmiņas ārējā). 
Izmeti izmanto programmu atkļūdošanai.

Instance of the program - Programmas instance - izpildāmās programmas kopija, kurai ir nepieciešama vieta operatīvajā atmiņā.

Chunk - Gabals - nepārtraukts atmiņas gabals ar noteikto struktūru.

ptmalloc2 - atvērtā pirmkoda programmatūra, kura nodrošina lietotāja līmeņa atmiņas iedalīšanu. Realizācija ptmalloc2 ir daļa no GNU C bibliotēkas, kura nodrošina dinamisko atmiņas iedalīšanu, izmantojot malloc(), free(), realloc() funkcijas izsaukumus.

dlmalloc (Doug Lea's Malloc) - atvērtā pirmkoda programmatūra, kura nodrošina lietotāja līmeņa atmiņas iedalīšanu, uz kuru balstīta ptmalloc/ptmalloc2/ptmalloc3 realizācijas.

bin - viensaišu vai dubultsaišu saraksts, kurā tiek uzglabāti atbrīvoti atmiņas gabali.

Memory leak - Atmiņas noplūde - ir problēma, kas notiek nepareizās lietotāja atmiņas pārvaldības dēļ, kad atmiņa, kura vairs netiks izmantota programmā, netiek atbrīvota.

