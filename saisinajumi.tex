\noindent 

POSIX (Portable Operating System Interface) - IEEE un ISO standartu kopa,  kas reglamentē kā rakstīt pieteikumu pirmkodu tā, lai pieteikumi būtu pārnēsājami starp operētājsistēmām.

IEEE (Institute of Electrical and Electronics Engineers) - Elektrotehnikas un elektronikas inženieru institūts.

ISO (International Organization for Standardization) - Starptautiskā Standartu organizācija.

Hard link - Stingrā saite - rādītājs uz datnes indeksa deskriptoru. 

Heap - Kaudze - atmiņas apgabals,  kurš tiek izmantots dinamiskajai atmiņas iedalīšanai.

ELF (Executable and Linkable Format) - ELF formāts -  bināro datņu formāts, kurš ir Unix un Linux standarts. Šīs formāts var būt izmantots priekš izpildāmam datnēm, objektu datnēm, bibliotēkām un atmiņas izmetēm.

Memory allocation - Atmiņas iedalīšana - atmiņās adreses piesaistīšana instrukcijām un datiem.

Static memory allocation - Statiskā atmiņas iedalīšana -  atmiņas iedalīšanas paņēmiens, kurš ir pielietots kompilācijas laikā.

Dynamic memory allocation - Dinamiskā atmiņas iedalīšana - atmiņas iedalīšanas paņēmiens, kurš pielietots programmas izpildes laikā. 

Core dump - Atmiņas izmete - visa atmiņas satura vai tā daļas pārrakstīšana citā vidē (parasti - no iekšējās atmiņas ārējā). 
Izmeti izmanto programmu atkļūdošanai.

Instance of the program - Programmas instance - izpildāmās programmas kopija, kurai ir nepieciešama vieta operatīvajā atmiņā.

Chunk - Gabals - nepārtraukts atmiņas gabals ar īpatnējo struktūru.

ptmalloc2 - atvērtā pirmkoda programmatūra. Realizācija ptmalloc2 ir daļa no GNU C bibliotēkas, kura nodrošina dinamisko atmiņas iedalīšanu, izmantojot malloc(), free(), realloc() funkcijas izsaukumus.

