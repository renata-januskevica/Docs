\noindent 

POSIX - IEEE un ISO standartu kopa, kurā ir aprakstīta saskarne starp programmām un operētājsistēmām.

Stingrā saite - rādītājs uz datnes indeksa deskriptoru. 

Kaudze - atmiņas apgabals,  kurš tiek izmantots dinamiskajai atmiņas iedalīšanai.

ELF -  bināro datņu formāts, kurš ir Unix un Linux standarts. Šīs formāts var būt izmantots priekš izpildāmam datnēm, objektu datnēm, bibliotēkām un atmiņas izmetēm.

Atmiņas iedalīšana - atmiņās adreses piesaistīšana instrukcijām un datiem.

Statiskā atmiņas iedalīšana -  atmiņas iedalīšanas paņēmiens, kurš ir pielietots kompilācijas laikā.

Dinamiskā atmiņas iedalīšana - atmiņas iedalīšanas paņēmiens, kurš pielietots programmas izpildes laikā. 

Atmiņas izmete - visa atmiņas satura vai tā daļas pārrakstīšana citā vidē (parasti - no iekšējās atmiņas ārējā). 
Izmeti izmanto programmu atkļūdošanas procesā.

Programmas instance - izpildāmās programmas kopija, kura tiek ierakstīta operatīvajā atmiņā.

Chunk - napartraukts atmiņas gabals ar īpatnejo struktūra.

TSD (Thread-Specific Data) - pavedienam specifiskie dati.
