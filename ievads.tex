Daudzas problēmas C un C++ programmās ir saistītas ar dinamisko atmiņu \cite{TFM}.
Problēmas ir grūti atkļūdot, jo parasti nav tiešo pazīmju, kas liecinātu par problēmām.
Tāpēc, lai saprastu, kas notiek, ir svarīgi zināt, kā realizēta atmiņas gabala iedalīšana un atmiņas pārvaldība no programmas un iedalītāja puses.
Taču programmētāju zināšanas bieži ir ierobežotas ar malloc() un free() funkciju izmantošanu. 
Tāpēc, lai atkļūdotu lietotnes, tiek veidotas speciālās uzturēšanas komandas.
Atkļūdošanas darbs nav viegls \cite{ER}. Parasti nav piekļuves klientu sistēmām vai piekļuve ir ierobežota, kā arī ļoti bieži sniegtā informācija par problēmu nav pietiekoša, lai varētu viennozīmīgi identificēt problēmu.
Var būt grūti atkārtot problēmu pat tad, ja ir aprakstīti scenāriji un ir pielikta konfigurācija.
Tas viss sarežģī atkļūdošanas procesu un padara darbu neefektīvu.

Šajā bakalaura darbā tiek aprakstīta un piedāvāta metode, kura palīdzes vienkāršot atkļūdošanas procesu problēmām, kas saistītas ar kaudzi un dinamisko atmiņas iedalīšanu.
Piemēram, kaudzes problēmas var būt šādas: atmiņas noplūde, datu kaudzes bojāšana, maksimālās atmiņas izmantošanas problēma, fragmentēšana, kļūdas trešās puses bibliotēkās.
Darbā ir aprakstīta atkļūdošanas metode, kura balstīta uz atmiņas izmetes analīzi.
Šī metode ir aprakstīta trijos izvēlētos problēmu piemēros: atmiņas noplūde, fragmentēšana, maksimālās atmiņas izmantošanas problēma.
Atmiņas noplūde ir izvēlēta, tāpēc ka tā ir viena no bieži sastopamām problēmām \cite{GNED}.
Fragmentēšana un maksimālās atmiņas izmantošanas problēma ir divas problēmas, kuras pēc autores viedokļa, ir tuvas, un var tikt atkļūdotas ar atmiņas izmetes palīdzību.
Metode var tikt pielietota gadījumos, kad ir novērojamas problēmu sekas vai, kad ir vēlme pārliecināties par problēmas eksistenci, sasniedzot kādu stāvokli programmā.
%Dotās problēmas demonstrē, ka pieeja strādā un var tikt izmantota plašāk, piemēram, citu problēmu atkļūdošanai.



Darbs sastāv no ievada, četrām nodaļām un secinājumiem:
\begin{itemize}
    \item Nodaļā "{\@nodone}" ir aplūkots atmiņas izmetes jēdziens, atmiņas izmetes ģenerēšanas un izmantošanas iespējas.
    \item Nodaļā "{\@nodtwo}" ir aprakstīta atmiņas organizācija un  ptmalloc2 realizācija. 
    \item Nodaļā "{\@nodthree}" ir pētītas trīs no piecām izvēlētām problēmām un identificētas to pazīmes atmiņas izmetē, nodaļā iekļauti arī pārējo divu problēmu apraksti.
    \item Nodaļā "{\@nodfour}" ir piedāvāta metode, kura balstīta uz iepriekš izklāstītiem jēdzieniem.
    \item Nodaļā "{\@nodsix}" ir apkopoti darba laikā gūtie rezultāti un secinājumi.
\end{itemize} 
