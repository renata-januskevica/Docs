\label{sect:Motivation}
\begin{itemize}
\item {\it Izkliedēta determinēta datu savākšana un komunikācija:} Tiek piedāvāts izmantot klasterizāciju informācijas plūsmu determinēšanai, kad, pateicoties specializētam MAC protokolam un klasteru uzstādījumiem, var panākt vēlamo informācijas plūsmu. Ka arī, mezglam nav jāuztur tabulas ar kaimiņu vai adresātu datiem. Rezultātā starp-klasteru komunikācijas īstenošana prasa mazāk resursu, jo starp klasteriem datus pārraidīs tikai CH, pārējie mezgli par citu klasteru eksistenci nezinās. CH eksistence nodrošina klastera vadību, proti, vadības komandas tiek nosūtītas tikai tai, nevis katram sensoru mezglam atsevišķi.
% \item {\it Determinēta datu savākšana:} Tiek piedāvāts izmantot klasterizāciju informācijas plūsmu determinēšanai, kad, pateicoties specializētam MAC protokolam un klasteru uzstādījumiem, var panākt vēlamo informācijas plūsmu.
% \item {\it Tīkla komunikācija pieprasa mazāk resursus:} tas tā ir tāpēc, ka mezglam nav jāuztur tabulas ar kaimiņu vai adresātu datiem; kā arī starp-klasteru komunikācijas īstenošana prasa mazāk resursu, jo starp klasteriem datus pārraidīs tikai CH, pārējie mezgli par citu klasteru eksistenci nezinās.
% \item {\it Tīkla pārvaldības vienkāršošana:} CH eksistence nodrošina klastera vadību, proti, vadības komandas tiek nosūtītas tikai tai, nevis katram sensoru mezglam atsevišķi.

\item {\it Sensoru mezglu grupēšana:} Klasteru sadalīšana ļauj sensoru mezglus apvienot pēc uzdevuma vai tā daļas efektīvākas komunikācijas sasniegšanai.
\item {\it Enerģijas patēriņa samazināšana:} Tas ir iepriekšējo divu punktu apvienojums. Datu paketes maršrutēšana no sensoru mezgla līdz notecei ir jāveic caur CH, rūpējoties tikai par dažu lēcienu maršrutēšanu. Šādu lēcienu skaitu definē veidojot klasteri, respektīvi, lēcienu skaits ir iepriekš uzstādīta vērtība, kā arī pakešu maršrutēšana pretējā virzienā (no noteces uz sensoru mezglu) notiek caur klastera CH. Tas ļauj ievērojami samazināt atsevišķi ņemta sensoru mezgla enerģijas patēriņu veicot datu pakešu maršrutēšanu. Apvienojot sensoru mezglus klasteros, samazinās sadursmju skaits, jo pastāv iespēja nodrošināt efektīvāku komunikāciju klastera ietvaros, nekā, ja mezgls nebūtu iedalīts klasterī. Klasterus sadala tā, lai tie pēc iespējas mazāk iespaido cits citu. Kā arī CH var organizēt komunikāciju klasterī tā, lai sensoru mezgli pārietu pagarinātā gulēšanas režīmā.
\item {\it Datu agregāciju:} Tā datu dzēšana klastera galvā, kas atkārtojas vai nav nepieciešami tālākajai pārsūtīšanai, dzēšana no tīkla, kas arī noved pie barošanas enerģijas ekonomijas un uzlabo datu pārsūtīšanai nepieciešamo laiku.
\item {\it Klasteru pašorganizācija:} Spēja pašiem sensoru mezgliem labot klastera struktūru gadījumā, ja kāds no tiem ir izgājis no ierindas.
\item {\it Viegla saprašana un izmantošana:} Klasterizācija ir uzskatāmāka lietotājam, jo sniedz iespēju loģiski sadalīt WSN grupās vai veikt sadalīšanu automātiskā režīmā pēc uzdotā uzdevuma.
\end{itemize} 

Par piedāvāto sistēmu un tās līmeņiem detalizētāk tiks stāstīts \ref{chap:chapter_4}. un \ref{chap:chapter_5}. nodaļās.


\begin{figure}[h]
\begin{center}
\includegraphics[scale=0.2]{System_Architecture_Big}
\end{center}
\caption{\textbf{\fontsize{11}{12}\selectfont {Divu līmeņu klasterizēta WSN arhitektūra}}}
\label{fig:System_Architecture}
\end{figure}

\begin{figure}[h]
\begin{center}
\includegraphics[scale=0.2]{System_Architecture_Big}
\end{center}
\caption{\textbf{\fontsize{11}{12}\selectfont {Divu līmeņu klasterizēta WSN arhitektūra}}}
\label{fig:System_Architecture1}
\end{figure}
Arhitektūra, {\ref{table:kysymys}} attēls {\ref{fig:System_Architecture}}, dalās divās loģiskās daļās. Pirmā daļa jeb pirmais sistēmas slānis ir klasterizēts WSN. Attēlā tas ir apzīmēts kā apaļi klasteri, kas sastāv no bezvadu sensoru mezgliem (SN) un klastera galvām (CH). Visi klastera sensoru mezgli ir vienādi pēc konstrukcijas un var atšķirties tikai ar konkrētā mezgla lomu klastera ietvaros. Šī īpašība gan padara sistēmu vienkāršāku. Pateicoties SN vienādajai konstrukcijai, kļūst iespējams vides piekļuvei izmantot vienotu MAC protokolu klasteru iekšienē. Attēlā \upshape \ref{fig:System_Architecture} vides piekļuves protokols ir apzīmēts kā WSN BS-MAC, kas atšifrējas kā Bez Sadursmju MAC protokols. Maršrutēšana klastera ietvaros tiek reducēta līdz viena lēciena komunikācijai, jeb katrs sensoru mezgls atrodas tiešā cita sensoru mezgla radio diapazonā. Klastera veidošana tiek panākta ar autonomu klasterizācija algoritmu. Tas ir iebūvēts BS-MAC protokolā, kurš tiek nosaukts kā uzlabots BS-MAC protokols un, kurš arī veic autonomu klasteru formēšanu bez sadursmju manierē. Šī algoritma izmantošana ievieš sistēmas pašārstēšanās funkciju. Piemēram, ja kāds SN izies no ierindas, tas tiks aizvietots ar citu SN no cita klastera vai ar jaunu SN mezglu. Katrs klasteris darbojas savā unikālā frekvences kanālā, lai būtu iespējams klasterus uzturēt tuvu vienu otram un joprojām atbalstīt bez sadursmju komunikāciju. Savukārt unikālo kanālu skaits ierobežo maksimālo klasteru skaitu.

Katrs sensoru mezglu klasteris var strādāt gan autonomi, gan arī kopā ar vārteju (GW). Kā jau tika minēts vārtejas atrodas otrajā piedāvātās sistēmas slānī. Tām ir divi komunikācijas virzieni jeb saskarnes: uz WSN un uz TCP/IP tīklu. Katra GW tiek pieslēgta savam sensoru mezglu klasterim un piedalās klastera datu savākšanā, apstrādē un pārraidīšanā TCP/IP tīklā līdz galvenajam datoram. Divas tīkla saskarnes esamība padara iespējamu sistēmas vieglu integrāciju eksistējošā TCP/IP infrastruktūrā. Attēlā \upshape \ref{fig:System_Architecture} arī ir parādītas komunikācijas saites, kas iespējamas starp sistēmas elementiem. Ar nepārtrauktu līniju ir parādīts galvenais datu ceļš sistēmā jeb tā komunikācija, kas ir pieejams, sākotnēji ieslēdzot sistēmas komponentes. Šajā gadījumā GW savāc datus no klastera un veic to pārraidīšanu TCP/IP tīklā ar domu nogādāt gala lietotājām. Savukārt saite apzīmēta ar raustītu līniju ir starp-vārteju saite, kuru izmanto ziņojumu maršrutēšanai starp GW un pa kuru vārtejas apmainās ar servisa informāciju. Šī saite tiek izveidota tikai pēc GW sazināšanās ar galveno datoru, no kura nolasa sistēmas stāvokļa tabulu, kas satur visu GW IP adreses. Kā ir redzams attēlā, šī saite izmanto TCP/IP protokolu steku. Un pēdējā sistēmas saite apzīmēta punkts-svītra saiti ir domāta papildus klasteru bez vārtejas atbalstīšanai. Un Šeit komunikācijas protokols ir BS-MAC, respektīvi, tāds pats kā GW, sazinoties ar savu tiešo sensoru mezglu klasteru. Proti, šīs iespējas realizācija paaugstina sistēmas robustumu, bet samazina katra klastera veiktspēju, galvenokārt, klasteru pārslēgšanās nepieciešamības dēļ. 

 

Stacionāru GW izmantošana kopā ar TDMA-bāzētu MAC protokolu sistēmu padara determinētu. Šī īpašība atklāj iespējas būvēt laika kritiskus lietojumus, kur ir jāgarantē datu savākšana un to apstrāde.

Par piedāvāto sistēmu un tās līmeņiem detalizētāk tiks stāstīts un nodaļās.

\begin{table}
\caption{\textbf{\fontsize{11}{12}\selectfont {\\ \LaTeX\ font selection}}} 
\label{table:kysymys}
\centering
	\begin{tabular}{|l|c|r|p{5cm}|}
	  \hline
	left & centered & right & a fully justified paragraph cell\\
	\hline
	  l & c & r & p\\
	  \hline
	\end{tabular}
\end{table}

\begin{table}
\caption{\textbf{\fontsize{11}{12}\selectfont {\\ \LaTeX\ font selection}}} 
\centering
	\begin{tabular}{|l|c|r|p{5cm}|}
	  \hline
	left & centered & right & a fully justified paragraph cell\\
	\hline
	  l & c & r & p\\
	  \hline
	\end{tabular}
\end{table}
Promocijas darba pētījuma priekšmets ir divu līmeņu klasterizēta bezvadu sensoru tīkla arhitektūra. Tajā skaitā arī WSN klasterizācija ar datu bez sadursmju pārraidīšana.
Promocijas darba pētījuma objekts ir komunikācijas protokolu un atbilstošu algoritmu steks, kas nodrošina pētījuma priekšmeta izveidošanu. Turklāt, pētījuma objekts tiek paplašināts ar jautājumiem, kas saistīti ar klasterizēta tīkla autonomu pārkonfigurēšanu, pašārstēšanu un uzstādīšanu. Kā arī pētāmās sistēmas jēdzienā ir iekļauti algoritmu un risinājumu pētīšana starp-klasteru komunikācijai jeb datu maršrutēšanai starp klasteriem un galveno datoru.


