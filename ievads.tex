Daudzas problēmas C un C++ programmās ir saistītas ar dinamisko atmiņu \cite{TFM}.
Problēmas ir grūti atkļūdot, jo nav tiešo pazīmju, kas liecinātu par problēmām.
Tāpēc lai saprastu, kas notiek, ir svarīgi zināt kā notiek atmiņas gabala iedalīšana un atmiņas pārvaldība no programmas un iedalītāja puses.
Taču programmētāju zināšanas bieži ir ierobežoti ar malloc() un free() funkciju izmantošanu. 
Tāpēc, lai atkļūdotu lietotnes tiek veidotas speciālās uzturēšanas komandas.
Atkļūdošanas darbs nav viegls \cite{ER}. Bieži nav piekļuves klientu sistēmām vai piekļuve ir ierobežota, kā arī ļoti bieži sniegtā informācija par problēmu nav pietiekoša, lai varētu viennozīmīgi identificēt problēmu.
Var būt grūti atkārtot problēmu pat tad, ja ir aprakstīti scenāriji un ir pielikta konfigurācija.
Tas viss sarežģī atkļūdošanas procesu un padara darbu neefektīvu.

Šajā bakalaura darbā tiks aprakstīta un piedāvāta pieeja, kura palīdzes vienkāršot atkļūdošanas procesu problēmām, kas saistītas ar kaudzi un dinamisko atmiņas iedalīšanu.
Piemēram: atmiņas noplūde, datu kaudzes bojāšana, maksimālās atmiņas izmantošanas problēma, fragmentēšana, kļūdas trešās puses bibliotēkās.
Darbā ir aprakstīta atkļūdošanas pieeja, kura balstīta uz atmiņas izmetes analīzi.
Šī pieeja ir aprakstīta uz 3 izvēlētiem problēmu piemēriem: atmiņas noplūde, fragmentēšana, maksimālās atmiņas izmantošanas problēma.
Atmiņas noplūde ir izvēlēta, tāpēc ka ir viena no bieži sastopamām problēmām.
Fragmentēšana un maksimālās atmiņas izmantošanas problēma ir divas problēmas, kuras pēc autores viedokļa, ir līdzīgas, un var tikt atkļūdotas ar atmiņas izmetes palīdzību.
Dotās problēmas demonstrē, ka pieeja strādā un var tikt izmantota plašāk, piemēram, citu problēmu atkļūdošanai.

Pieeja, kas aprakstīta bakalaura darbā, tiek izstrādāta ievērojot sekojošus ierobežojumus:
\begin{itemize}
	\item GNU C bibliotēkas izmantošana sākot ar 2.3 versiju;
    \item a.out vai ELF atmiņas izmetes formāts;
    \item ar atmiņu saistīto problēmu atkļūdošana.
\end{itemize} 

Darba struktūra:
\begin{enumerate}
    \item   nodaļā ietverts atmiņas izmetes un atmiņas organizācijas apraksts;
    \item   nodaļā aprakstītas problēmas un identificētas tās pazīmes atmiņas izmetē;
    \item   nodaļā piedāvāta metode, kura balstīta uz atmiņas izmeti;
    \item   nodaļā "Galvenie rezultāti un secinājumi" apkopotie darba laikā gūtie rezultāti un secinājumi.
\end{enumerate} 
